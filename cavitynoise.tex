%\newpage
\section{Derivation of the Function that Models the Noise Spectrum} 
\label{sec:cavitynoise}

The background noise from a cavity is governed by the thermal noise and the 
vacuum fluctuation. %from the cavity. 
According to Planck's law in one dimension (1D), the spectral density of 
the electromagnetic noise from the cavity, thermalized with an environment of 
temperature $T_{\rm c}$, through a transmission line is 
\begin{equation}
\label{eq:cavity_thermal_spectral_density}
    S(\omega) = \hbar\omega \left( \frac{1}{e^{\hbar\omega/k_{\rm B}T_{\rm c}} - 1} +\frac{1}{2} \right),
\end{equation}
where $\omega$ is the angular frequency, $\hbar$ is the reduced Planck's 
constant, and $k_{\rm B}$ is the Boltzmann constant. 
%The first and the second 
%terms in Eq.~\eqref{eq:cavity_thermal_spectral_density} are the thermal noise 
%and the vacuum fluctuation from the cavity, respectively. The spectrum is 
%white when considering a narrow frequency span.

However, the cavity body (the materials that form the cavity itself) may not 
be thermalized with its 1D electromagnetic environment. To understand the 
noise spectrum from the cavity near its 
resonant frequency $\omega_{\rm c}/2\pi$ in this scenario, the model in 
Fig.~\ref{fig:cavity_in_out} is considered. Through a probe the cavity field 
mode $c$ is coupled to the modes $a_2$ of a 1D transmission line, representing
 the path toward a signal receiver, with a rate $\kappa_2$. The cavity field 
is also coupled to the modes of the cavity body $a_0$, representing the 
intrinsic loss, with a rate $\kappa_0$. In a steady state, 
the quantum input-output theory leads to a relation between the outgoing field 
from the cavity to the 1D transmission line, $a_{\rm 2,out}$, and the incoming
 fields, $a_{\rm 2,in}$ and $a_{\rm 0,in}$, through the elements of the 
cavity scattering matrix:
\begin{equation} \label{eq:outgoing_field}
	 a_{\rm 2,out} = S^{*}_{\rm 22} a_{\rm 2,in} + S^{*}_{\rm 20} a_{\rm 0,in},
\end{equation}
where $S_{\rm 22} = \frac {\kappa_0 - \kappa_{\rm 2} + i 2 \Delta} {\kappa_0+\kappa_{\rm 2} + i 2 \Delta}$, $S_{\rm 20} = \frac {2\sqrt{\kappa_0 \kappa_{\rm 2}}} {\kappa_0+\kappa_{\rm 2} + i 2 \Delta}$,
and $\Delta = \omega-\omega_{\rm c}$ is the detuning. 

As both incoming fields are in a thermal state, $\langle a_{i,\rm in}(0) a_{i,\rm in}(\tau) \rangle = n_{\rm th}(T_i) \delta(\tau)$, where $n_{\rm th}(T_i) = \frac{1}{e^{\hbar\omega/k_{\rm B}T_i} - 1}$ is the mean thermal photon number 
of the incoming field $a_{i,\rm in}$ at the temperature $T_i$, and 
$\delta(\tau)$ is the $\delta$-function. In the model the incoming field 
$a_{\rm 2,in}$ comes from a nearby attenuator, anchored to the mixing flange, 
in the transmission line 
with a temperature $T_{\text mx} \equiv T_{\rm 2}$, and $a_{\rm 0,in}$ comes 
from the cavity body with a temperature $T_{\text c} \equiv T_{\rm 0}$.

By defining the effective temperature $\Tilde{T}_i = \left( n_{\rm th}(T_i) + \frac{1}{2} \right) \hbar\omega/k_{\rm B}$,
the power spectral density of the outgoing field of the transmission line 
modes is
\begin{equation}
\label{eq:spectrum_relation}
\begin{aligned}
    S_{\rm out}(\omega) 
    =& \int_{-\infty}^{\infty} \hbar\omega \left(\langle a_{2,\rm out}(0) a_{2,\rm out}(\tau) \rangle + \frac{1}{2} \right) e^{-i\omega\tau} d\tau \\
    =& |S_{\rm 22}|^2 k_{\rm B} \Tilde{T}_2 + |S_{\rm 20}|^2 k_{\rm B} \Tilde{T}_0.
\end{aligned}
\end{equation}
%The total output noise can be viewed as the sum of the reflection of the 
%incoming noise from the attenuator by the cavity and the transmission of the 
%noise from the cavity body through the cavity. 
The total output noise can be viewed as the sum of the reflection of the 
incoming noise from the attenuator and the transmission of the noise from the 
cavity body itself.
Via the unitary property of 
the cavity scattering matrix, i.e. $|S_{\rm 22}|^2+|S_{\rm 20}|^2=1$,
\begin{equation}
\label{eq:spectrum_relation_Lorentz}
\begin{aligned}
    S_{\rm out}(\omega) =  k_{\rm B} \Tilde{T}_2 + k_{\rm B} (\Tilde{T}_0 - \Tilde{T}_2) L(\omega),
\end{aligned}
\end{equation}
where $L(\omega) = |S_{\rm 20}|^2 = \frac {\kappa_0 \kappa_{\rm 2}}  {(\kappa_0+\kappa_{\rm 2})^2/4 + \Delta^2}$ is a Lorentzian function with a FWHM 
$\kappa_0+\kappa_{\rm 2}$. Therefore, the noise spectrum has a flat background
 determined by the incoming noise of the attenuator with an effective 
temperature $\Tilde{T}_{\rm 2}$, plus an excess Lorentzian peak centered at 
$\omega_{\rm c}$ determined by the effective temperature difference 
$\Tilde{T}_{\rm 0} - \Tilde{T}_{\rm 2}$. (The center Lorentzian structure 
can even be a dip if ${T}_{\rm 0} < {T}_{\rm 2}$.)

\begin{figure}[htbp]
    \centering
    \includegraphics[width=8.6cm]{figures/inout.png}
    \caption{The input-output model of cavities. The cavity field mode $c$ is 
coupled to the modes of a 1D transmission line $a_2$ (with a rate $\kappa_2$)
 and the modes of the cavity body $a_0$ (with a rate $\kappa_0$). 
The incoming and outgoing fields of the transmission line 
are separated by the circulator. The attenuator and the cavity body emitting 
the fields $a_{2,\rm in}$ and $a_{0,\rm in}$ are thermalized at $T_{\rm mx}$ 
and $T_{\rm c}$, respectively. 
}
    \label{fig:cavity_in_out}
\end{figure}



