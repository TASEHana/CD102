\section{Analysis of the Synthetic Axion Data}\label{sec:faxion}
After TASEH finished collecting the CD102 data on November 15, 2021, 
the synthetic axion signals were injected into the cavity and read out via the 
same transmission line and amplification chain. The procedure 
to generate axion-like signals is summarized in 
Ref.~\cite{TASEHInstrumentation}. 
Due to the uncertainties on the losses of signal transmission
 lines, the synthetic axion signals are not used to perform an absolute 
calibration of the search sensitivity. Instead, 
a test with synthetic axion signals could be used to verify the procedures of 
data acquisition and physics analysis. The 
SNR of the frequency bin with maximum power from the 
synthetic axion signals, at 4.708970~GHz, was set to $\approx 3.35$.%, 
%corresponding to a power of $\approx 6.03 \times 10^{-13}$~W in a 1-kHz 
%frequency bin.  

The same analysis procedure as described in Sec.~\ref{sec:ana} is applied 
to the data with synthetic axion signals. 
Figure~\ref{fig:faxionstep} presents the individual raw power spectra in 
the 24 frequency scans. Before combining 
the 24 spectra, the SNR of the maximum-power bin is measured to be 
3.577. %the SNR is slightly higher than 3.35 due to a 
%5\% difference in the noise fluctuation between the measurements from 
%the calibration and the measurements taken right before injecting 
%axion-like signals. 
After the combination of the spectra and the merging of five frequency 
bins, the SNRs increase to 4.74 and 6.12, respectively. In addition to the 
injected synthetic axion signal, a candidate at 4.708006~GHz is found after 
merging the spectra. Due to the limited access to the DR, 
it is not possible to perform a rescan after the collection of the 
synthetic axion data. Therefore,  
the CD102 data from the two scans that had resonant frequencies close to 
the candidate frequency are added so to mimic the rescan; the candidate is 
found to be a statistical fluctuation.  
%Figures~\ref{fig:faxioncombine}--~\ref{fig:faxionmerge} present 
%the RDP spectra with the corresponding SNR, respectively, after combining 
Figure~\ref{fig:faxioncombinemerge} presents 
the SNR %after combining the spectra that share the same frequency bins and 
%after merging five neighboring bins, respectively; 
after the combination and the merging, respectively;    
the 24 scans of the 
synthetic axion data and the two 
scans of the CD102 data are included and processed together. 
The analysis results of the synthetic axion signals prove that a power 
excess of more than 5$\sigma$ can be found at the expected frequencies via 
the standard analysis procedure.  

\begin{figure}[htbp]                                                                                                  
    \centering                                                                                                                       
    \includegraphics[width=8.6cm]{figures/faxion_rawpower_24steps.png}
 \caption{The raw output power spectra, before applying the 
 SG filter, from the 24 frequency steps of the synthetic axion 
data. In order to show the spectra clearly, the spectra are shifted 
with respect to each other with an arbitrary offset in the vertical scale.}                
\label{fig:faxionstep}                                                                                                            
\end{figure}                       

%\begin{figure}[htbp]                                                                                                  
%    \centering                                                                                                                       
%    \includegraphics[width=8.6cm]{figures/Power_CombSpectrum_FaxionRun_AllSteps_Rescan_SG4_W201.png}
%    \includegraphics[width=8.6cm]{figures/SNR_CombSpectrum_FaxionRun_AllSteps_Rescan_SG4_W201.png}
%    \caption{The RDP (upper) and the 
%signal-to-noise ratio (lower) after combining the spectra 
%of the synthetic axion data with overlapping frequencies from different scans.
% The procedure and the weights for combination are summarized in 
%Sec.~\ref{sec:weighting_algorithm}.}                
%\label{fig:faxioncombine} 
%\end{figure}                       


%\begin{figure}[htbp]                                                   
%\centering                                                                                                                       
%   \includegraphics[width=8.6cm]{figures/Power_GrandSpectrum_FaxionRun_AllSteps_Rescan_Merged_5bin_SG4_W201_LqWeight.png}
%    \includegraphics[width=8.6cm]{figures/SNR_GrandSpectrum_FaxionRun_AllSteps_Rescan_Merged_5bin_SG4_W201_LqWeight.png}
%    \caption{The RDP (upper) and the signal-to-noise ratio (lower) after 
%merging the RDP measured in five neighboring frequency bins of the 
%synthetic axion data. 
%The procedure and the weights for merging 
%are summarized in Sec.~\ref{sec:merge}.}                
%\label{fig:faxionmerge}    
%\end{figure}                       



\begin{figure}[htbp]                                                                                                  
    \centering                                                                                                                       
    \includegraphics[width=8.6cm]{figures/SNR_CombSpectrum_FaxionRun_AllSteps_Rescan_SG4_W201.png}
    \includegraphics[width=8.6cm]{figures/SNR_GrandSpectrum_FaxionRun_AllSteps_Rescan_Merged_5bin_SG4_W201_LqWeight.png}
    \caption{The signal-to-noise ratio, from the synthetic axion data, 
after combining the spectra with overlapping frequencies from different 
scans (upper) and after merging the RDP measured in five neighboring 
frequency bins (lower). 
The procedure and the weights for combination and merging are summarized in 
Sec.~\ref{sec:weighting_algorithm} and Sec.~\ref{sec:merge}, respectively.}               
\label{fig:faxioncombinemerge} 
\end{figure}                       

   
