\section{Analysis of the Synthetic Axion Data}\label{sec:faxion}
After TASEH finished collecting the CD102 data on November 15, 2021, 
the synthetic axion signals were injected into the cavity and read out via the 
same transmission line and amplification chain. The procedure 
to generate axion-like signals is summarized in 
Ref.~\cite{TASEHInstrumentation}. 
Due to the uncertainties on the losses of readout electronics and transmission
 lines, the synthetic axion signals were not used to perform an absolute 
calibration of the search sensitivity. Instead, 
a test with synthetic axion signals could be used to verify the procedures of 
data acquisition and physics analysis. The 
signal-to-noise ratio (SNR) of the frequency bin with maximum power from the 
synthetic axion signals, at 4.708970~GHz, was set to $\approx 3.35\sigma$, 
corresponding to a power of $\approx 6.03 \times 10^{-13}$~W in a 1-kHz 
frequency bin.  

The same analysis procedure as described in Section~\ref{sec:ana} was applied 
to the data with synthetic axion signals. 
Figure~\ref{fig:faxionstep} presents the individual raw power spectra in 
24 frequency scans. Before combining 
the 24 spectra vertically, the SNR of the maximum-power bin is measured to be 
3.577$\sigma$; the SNR is slightly higher than 3.35$\sigma$ due to a 
5\% difference in the noise fluctuation between the measurements from 
the calibration and the measurements taken 
right before injecting axion-like signals. After the vertical combination 
of power spectra and the merging of five frequency bins, the SNRs increase to 
4.74$\sigma$ and 6.12$\sigma$, respectively. In addition to the 
injected synthetic axion signal, a candidate at 4.708006~GHz was found after 
merging the spectra. Since it was not possible to perform a rescan later, 
the real axion data from the two scans that had resonant frequencies close to 
the candidate frequency were added so to mimic the rescan; the candidate 
 disappeared afterwards and is a statistical fluctuation.  
Figures~\ref{fig:faxioncombine}--~\ref{fig:faxionmerge} present 
the spectra with the corresponding SNR, respectively, after combining the 24 
spectra vertically and after merging five neighboring bins, 
including both the 24 scans of the synthetic axion data and the two scans 
of the real axion data. 
The analysis results of the synthetic axion signals prove that an power 
excess of more than 5$\sigma$ can be found at the expected frequencies via 
the standard analysis procedure.  

\begin{figure}[htbp]                                                                                                  
    \centering                                                                                                                       
%    \includegraphics[width=0.48\textwidth]{figures/RawSpectra_Faxion_YAxis_Shifted.pdf}
    \includegraphics[width=8.6cm]{figures/faxion_rawpower_24steps.png}
 \caption{The raw output power spectra, before applying the 
 SG filter, from the 24 frequency steps of the synthetic axion 
data. In order to show the spectra clearly, the spectra are shifted 
with respect to each other with an arbitrary offset in the vertical scale.}                
\label{fig:faxionstep}                                                                                                            
\end{figure}                       

\begin{figure}[htbp]                                                                                                  
    \centering                                                                                                                       
    \includegraphics[width=8.6cm]{figures/Power_CombSpectrum_FaxionRun_AllSteps_Rescan_SG4_W201.png}
    \includegraphics[width=8.6cm]{figures/SNR_CombSpectrum_FaxionRun_AllSteps_Rescan_SG4_W201.png}
    \caption{The power (upper) and the 
signal-to-noise ratio (lower) after combining the spectra 
of the synthetic axion data
with overlapping 
frequencies vertically. The procedure and the weights for combination 
are summarized in Section~\ref{sec:ana}.}                
\label{fig:faxioncombine}                                                                                                            
\end{figure}                       


\begin{figure}[htbp]                                                                                                  
    \centering                                                                                                                       
    \includegraphics[width=8.6cm]{figures/Power_GrandSpectrum_FaxionRun_AllSteps_Rescan_Merged_5bin_SG4_W201_LqWeight.png}
    \includegraphics[width=8.6cm]{figures/SNR_GrandSpectrum_FaxionRun_AllSteps_Rescan_Merged_5bin_SG4_W201_LqWeight.png}
    \caption{The power (upper) and the signal-to-noise ratio (lower) after 
merging the power measured in five neighboring frequency bins of the 
synthetic axion data. 
The procedure and the weights for merging 
are summarized in Section~\ref{sec:ana}.}                
\label{fig:faxionmerge}    
\end{figure}                       

   
