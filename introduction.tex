%%%%%%%%%%%%%%%%%%%%%%%%%%%%%%%%%%%%%%%%%%%%%%%%%%%%%%%%%%%%%%%%%%%%%%%%%%%%%%%
\section{Introduction} \label{sec:intro}
The axion is a hypothetical particle predicted as a consequence of a  
solution to the strong CP problem~\cite{strongCPI,strongCPII,strongCPIII}, 
i.e. why the CP symmetry is conserved in the strong 
interactions when there is an explicit CP-violating term in the QCD 
Lagrangian. In other words, why is the electric dipole moment 
of the neutron so tiny:  
%$d_n =\left(0.0\pm 1.1_\mathrm{stat} \pm 0.2_\mathrm{sys}\right)\times10^{-26}~e\cdot\mathrm{cm}$~\cite{EDM}? 
$\left|d_n\right| < 1.8 \times10^{-26}~e\cdot\mathrm{cm}$~\cite{EDM,PDG}? 
The solution proposed by Peccei and Quinn is to introduce a new global 
Peccei-Quinn U(1)$_\mathrm{PQ}$ symmetry that is spontaneously broken; the 
axion is the pseudo Nambu-Goldstone boson of 
U(1)$_\mathrm{PQ}$~\cite{strongCPI}. 
Axions are abundantly produced during the QCD phase transition in 
the early universe and may constitute the dark matter (DM). 
%Refs~\cite{axionDMI,axionDMII,axionDMIII} also suggest that axions form a 
%Bose-Einstein condensate; this property explains the occurrence of 
%caustic rings in galactic halos. 
In the post-inflationary PQ symmetry breaking scenario, where the PQ symmetry
is broken after inflation, current calculations suggest a mass range of 
${\cal O}(1–100)$~\muevcc\ for axions so that the cosmic axion density does 
not exceed the 
observed cold DM density~\cite{QCDCalI,QCDCalII,QCDCalIII,QCDCalIV,QCDCalV,QCDCalVI,QCDCalVII,QCDCalVIII,QCDCalIX,QCDCalX,QCDCalXI,QCDCalXII,QCDCalXIII}. 
Therefore, axions are compelling because they may explain at the same 
time two puzzles that are on scales different by more than thirty orders of 
magnitude. 


%
%
%

%However, in the pre-inflationary PQ symmetry breaking scenario, where
%the PQ symmetry is broken before and during inflation and not restored
%afterwards, $m_a$ could have a much smaller value~\cite{PDG}. 
%
%
%

Axions could be detected and studied via their two-photon interaction, the
so-called ``inverse Primakoff effect''. For QCD axions, i.e. the axions 
proposed to solve the strong CP problem, the axion-two-photon coupling 
constant \bgagg\ is related to the mass of the axion \ma: 
\begin{equation}
 \bgagg = \left(\frac{\bggamma\alpha}{\pi \Lambda^2}\right)\ma, 
\label{eq:grelation}
\end{equation}
where \bggamma\ is a dimensionless model-dependent parameter, $\alpha$ is the 
fine-structure constant, $\Lambda=78~\MeV$ is a scale parameter that can 
be derived from the mass and the decay constant of the pion and the ratio of 
the up to down quark masses. 
The numerical values of \bggamma\ are -0.97 and 0.36 
in the Kim-Shifman-Vainshtein-Zakharov (KSVZ)~\cite{KSVZI,KSVZII} and 
the Dine-Fischler-Srednicki-Zhitnitsky (DFSZ)~\cite{DFSZI,DFSZII} benchmark 
models, respectively. 
%
%
%Apart from the QCD axions, one could still consider a more general 
%case of axion-like particles (ALPs) where the two parameters \gagg\ 
%and \ma\ are independent from each other 
%and search for ALPs via the two-photon interaction. 
% 

The detectors with the best sensitivities to axions with a mass of 
$\approx \muevcc$, as first put forward by 
Sikivie~\cite{SikivieI,SikivieII},  
are haloscopes consisting of a microwave cavity immersed in a strong static 
magnetic field and operated at a cryogenic temperature. 
%In the presence of an external magnetic field, the ambient 
%oscillating axion field induces an electromagnetic field that oscillates with 
%a microwave frequency $f$ set by the total energy of the axion: 
%$hf=E_a=\ma c^2 + \frac{1}{2}\ma v^2$. The induced electromagnetic field  
%resonantly drives 
%the electromagnetic modes in the cavity with a resonant frequency $f$; 
In the presence of an external magnetic field, the ambient oscillating axion 
field drives the cavity and they resonate when the frequencies of the 
electromagnetic 
modes in the cavity match the microwave frequency $f$, where $f$ is set by 
the total energy of the axion: $hf=E_a=\ma c^2 + \frac{1}{2}\ma v^2$; the 
axion signal power is further delivered %in the form of microwave photons and 
%readout with a low-noise amplifier. 
to the readout probe followed by a low-noise linear amplifier. 
The axion mass is unknown, therefore, 
the cavity resonator must allow the possibility to be tuned through a range
of possible axion masses. The Axion Dark Matter eXperiment (ADMX), 
one of the flagship dark matter search experiments, had developed and 
improved the cavity design and readout electronics over the years. 
The results from the previous 
versions of ADMX and the Generation 2 ADMX (ADMX G2) excluded the KSVZ 
benchmark model within the mass range of %1.9--3.69~$\mu$eV$/c^2$. 
1.9--4.2\muevcc\ and the DFSZ benchmark model for the mass ranges 
of 2.66--3.31 and 3.9--4.1\muevcc, 
respectively~\cite{ADMXI,ADMXII,ADMXIII,ADMXIV,ADMXV,ADMXVI,ADMXVII}. 
%In addition to having sensitivity to the more weakly coupled DFSZ axions, 
One of the major goals of ADMX G2 is to search for higher-mass axions in the 
range of 4--40\muevcc\ (1--10~GHz), which is also the aim of  
the new haloscope experiments established during the last ten years.  
The Haloscope at Yale Sensitive to Axion Cold dark matter 
(HAYSTAC) had performed searches first for the mass range of 
23.15--24\muevcc~\cite{HAYSTACIII,HAYSTACIV} and later at 
around 17\muevcc~\cite{HAYSTACI}; they excluded axions 
with $\ggamma\geq 1.38 \ggamma^\text{KSVZ}$ for $m_a=16.96-17.12$ and 
17.14--17.28\muevcc~\cite{HAYSTACI}. The Center 
for Axion and Precision Physics Research (CAPP) constructed 
and ran simultaneously several experiments targeting at 
different frequencies~\cite{CAPPII,CAPPIII,CAPPI}; 
they have pushed the limits towards the KSVZ value within a narrow mass 
region of 10.7126--10.7186\muevcc~\cite{CAPPI}.
The QUest for AXions-$a\gamma$ (QUAX-$a\gamma$) also pushed their limits 
close to the upper bound of the QCD axion-two-photon couplings for 
$m_a\approx43\muevcc$~\cite{QUAX}.   

%This paper presents the first results and the analysis 
This paper presents the analysis 
details of a search for axions for the mass range of \mlo--\mhi\muevcc, 
from the Taiwan Axion Search Experiment with Haloscope (TASEH). 
The expected axion signal power and signal line shape, the noise power, 
and the signal-to-noise ratio are described in 
Secs.~\ref{sec:introsignal}--\ref{sec:intronoise}. An overview 
of the TASEH experimental setup is presented in Sec.~\ref{sec:taseh}. 
Section~\ref{sec:calibration} gives a brief 
description of the calibration for the whole amplification chain 
while Sec.~\ref{sec:ana} details the analysis procedure.  
Section~\ref{sec:faxion} presents the analysis of the 
synthetic axion 
data and Sec.~\ref{sec:sys} discusses the systematic 
uncertainties that may affect the limits on the \gagg. 
The final results and the conclusion are presented in 
Sec.~\ref{sec:results} and Sec.~\ref{sec:conclusion}, 
respectively. 


%%%%%%%%%%%%%%%%%%%%%%%%%%%%%%%%%%%%%%%%%%%%%%%%%%%%%%%%%%%%%%%%%%%
\subsection{The expected axion signal power and signal line shape}
\label{sec:introsignal}

The signal power extracted from a microwave cavity on resonance is given 
by~\cite{HAYSTACIII}:
\begin{equation}
P_s = \left(\bggamma^2\frac{\alpha^2\hbar^3c^3\rho_a}{\pi^2\Lambda^4}\right)\times
\left(\omega_c\frac{1}{\mu_0}B_0^2VC_{mnl}Q_L\frac{\beta}{1+\beta}\right),
\label{eq:ps}
\end{equation}
where $\rho_a=0.45~\GeV/\mathrm{cm}^3$ is the local dark-matter 
density. 
%density~\cite{[{}][{. 
Both $0.45~\GeV/\mathrm{cm}^3$ (used by ADMX, HAYSTAC, 
CAPP, and QUAX) and $0.3~\GeV/\mathrm{cm}^3$ (more commonly cited 
by the other direct DM search experiments) are consistent with the recent 
measurements~\cite{Read:2014qva,PDG}. 
%}]Read:2014qva,PDG}. 
The second set of parentheses contains parameters related to the experimental 
setup: the angular resonant frequency of the cavity $\omega_c$, 
the vacuum permeability $\mu_0$, the nominal strength of the external magnetic 
field $B_0$, the volume of the cavity $V$, and the loaded quality factor of 
the cavity 
\(Q_L=Q_0/(1+\beta)\), where $Q_0$ is the unloaded, intrinsic quality factor 
of the cavity and $\beta$ is the coupling coefficient which determines the 
amount 
of coupling of the signal to the receiver. The form factor $C_{mnl}$ is the 
normalized overlap of the electric field 
$\vec{\bm{E}}$, for a particular cavity resonant mode, with the external magnetic 
field $\vec{\bm{B}}$:
\begin{equation}
  C_{mnl} = \frac{\left[\int\left( \vec{\bm{B}}\cdot\vec{\bm{E}}_{mnl}\right) d^3\bm{x}\right]^2}{B_0^2V\int E_{mnl}^2 d^3\bm{x}}.
\label{eq:formfactor} 
\end{equation} 
The magnetic field $\vec{\bm{B}}$ in TASEH points mostly along the axial 
direction ($z$-axis) of the cavity. 
The field strength has a small variation along the radial and axial 
directions and $B_0$ is the nominal magnetic field strength. 
For cylindrical cavities, the largest form factor is from the 
TM$_{010}$ mode. The expected signal power derived from the experimental 
parameters of TASEH (see Table~\ref{tab:tasehbenchmark}) 
is $P_s\simeq 1.4\times10^{-24}$~W for a KSVZ axion with a 
mass of 19.5\muevcc. 

In the direct dark matter search experiments, several assumptions are 
made in order to derive a signal line shape. 
The density and the velocity distributions of DM are related to each other 
through the gravitational potential. The DM in the galactic halo is assumed 
to be virialized. The DM halo density distribution is assumed 
to be spherically symmetric and close to be isothermal, which results in a 
velocity distribution similar to the Maxwell-Boltzmann distribution. The 
distribution of the measured signal frequency can be further derived from the 
velocity distribution after a change of variables and set 
\(hf_a = \ma c^2\). 
%Previous experimental results typically adopt the 
%following function for frequency $f\ge f_a$: 
For frequency $f\ge f_a$:
\begin{equation}
\mathcal{F}(f, f_a) = \frac{2}{\sqrt{\pi}}\sqrt{f-f_a}\left(\frac{3}{\alpha}\right)^{3/2}
e^{\frac{-3\left(f-f_a\right)}{\alpha}}, 
\label{eq:simplesignal}
\end{equation}
where $\alpha\equiv  f_a \left<v^2\right>/c^2$. Previous axion searches 
typically adopt Eq.~\eqref{eq:simplesignal} when deriving their analysis 
results~\cite{HAYSTACII}. For a Maxwell-Boltzmann velocity 
distribution, the variance $\left<v^2\right>$ and the most probable velocity 
(speed) $v_p$ are related to each other:
$\left<v^2\right>=3v_p^2/2=$(270~km/s)$^2$, where $v_p=220$~km/s is the local 
circular velocity of DM in the galactic rest frame and this value is also 
used by other axion experiments. 

Equation~\eqref{eq:simplesignal} 
is modified if one considers that the relative velocity of the DM halo with 
respect to the Earth is not the same as the DM velocity in the galactic rest 
frame~\cite{SignalLineShapeI}. The velocity distributions shall also be 
truncated so that the DM velocity is not larger than the escape velocity of 
the Milky Way~\cite{Lisanti:2016jxe}. 
%Several N-body simulations~\cite{Diemand:2008in,Springel:2008cc} follow 
%structure formation from the initial DM density perturbations to the largest 
%halo today and take into account the merger history of the Milky Way, rather 
%than assuming that the Milky Way is in a steady state; the simulated results 
% suggest velocity distributions with more high-speed particles relative 
%to the Maxwellian case~\cite{Navarro:1995iw,Burkert:1995yz}. However, these 
%numerical simulations contain only DM particles; an inclusion of baryons may 
%enhance the halo's central density due to a condensation of gas towards the 
%center of the halo via an adiabatic 
%contraction~\cite{Blumenthal:1985qy,Gnedin:2004cx}, or may reduce 
%the density due to the supernova outflows, 
%etc~\cite{Mashchenko:2007jp,Governato:2009bg}. 
Several numerical simulations follow structure formation from the initial DM 
density perturbations to the largest halo today and take into account the 
merger history of the Milky Way, rather than assuming that the Milky Way is 
in a steady state. 
Earlier high-resolution DM-only simulations suggested velocity 
distributions noticeably different from the Maxwellian 
one~\cite{PDG,Lisanti:2016jxe,Green:2017odb}. 
The recent hydrodynamical simulations including baryons, 
which have a non-negligible effect on the DM distribution 
in the Solar neighborhood, find that 
the velocity distributions are closer to Maxwellian than 
 previously thought~\cite{PDG,Green:2017odb}. However, there may still be 
deviations and significant variations depending on the detailed 
characteristics of the halos. By studying the motion of stars that 
are expected to have the same kinematics as the DM, 
one could determine the DM velocity distribution from observations. 
The data from the Gaia satellite~\cite{GAIA} imply that the local DM halo, 
similar to the local stellar halo, may have 
a component that is quasi-spherical and a component that is radially anisotropic, 
giving a velocity distribution slightly shifted towards higher values 
with respect to the Maxwellian one~\cite{Evans:2018bqy}.  

In order to compare the results of TASEH with those of the former axion searches, 
the analysis presented in this paper uses the axion 
%signal line shape by including Eq.~\eqref{eq:simplesignal} in the weights 
%when merging the measured power from multiple frequency bins 
%(see Sec.~\ref{sec:merge}). A signal line width 
signal line shape from Eq.~\eqref{eq:simplesignal} (see Sec.~\ref{sec:merge}). 
A signal line width 
$\Delta f_a=\ma\left<v^2\right>/h\simeq$~5~kHz, which is much smaller than 
the TASEH cavity line width $f_a/Q_L\simeq$~250~kHz, is assumed. %and 
%five frequency bins are merged to perform the final analysis. 
For a signal 
line shape as described in Eq.~\eqref{eq:simplesignal}, a 5-kHz bandwidth 
includes about 95\% of the distribution. 
Still given the caveats above and a lack of 
strong evidence for any particular choice of the velocity distribution, 
two different scenarios are considered and their results are 
presented for comparison: (i) without an assumption of signal line shape, and 
(ii) assuming a Gaussian signal line shape with a narrower full width at half 
maximum (FWHM), see Sec.~\ref{sec:results} for more details. 
%In addition, a signal line width 
%$\Delta f_a=\ma\left<v^2\right>/h\simeq$~5~kHz, which is much smaller than 
%the TASEH cavity line width $f_a/Q_L\simeq$~250~kHz, is assumed and 
%five frequency bins are merged to perform the final analysis. For a signal 
%line shape as described in Eq.~\eqref{eq:simplesignal}, a 5-kHz bandwidth 
%includes about 95\% of the distribution.


 
\subsection{The expected noise and the signal-to-noise ratio}
\label{sec:intronoise}
Several physics processes can contribute to the total noise and all of them 
can be seen as Johnson thermal noise at some effective temperature, or the 
so-called system noise temperature \tsys. The total noise power in a 
bandwidth $\Delta f$ is then:
\begin{equation}
  P_n = k_B\tsys \Delta f, 
\end{equation}
where $k_B$ is the Boltzmann constant. 
%The system noise temperature \tsys\ has three major components: 
The system noise temperature \tsys\ has two major components: 
\begin{equation}
%  k_B\tsys = hf\left(\frac{1}{e^{\left.hf\right/k_{B}T_\mathrm{cavity}}-1} + \frac{1}{2} \right) + k_B\ta. 
%  \tsys = T_\text{bb} + T_\text{qn}  + \ta,
  \tsys = T_\text{cn}  + \ta,
\label{eq:pn}
\end{equation}
%where 
%\begin{equation}
%   T_\text{qn} = \frac{1}{2}\left. hf\middle/k_B\right..
%\end{equation}
% The three terms in Eq.~\eqref{eq:pn} correspond to the effective 
%temperatures of the following noise sources: (i) 
 The two terms in Eq.~\eqref{eq:pn} correspond to the effective 
temperatures of the following noise sources: %(i) 
%$T_\text{bb}=\left(\frac{1}{e^{\left.hf\middle/k_BT_\mathrm{c}\right.}-1}\right)\left. hf\middle/k_B\right.$, the blackbody 
%radiation from the cavity at a physical temperature $T_\mathrm{c}$, (ii) 
%$T_\text{qn} = \frac{1}{2}\left. hf\middle/k_B\right.$, the quantum noise 
%associated with the zero-point fluctuation of the vacuum, 
(i) 
$T_\text{cn}=\left(\frac{1}{e^{\left.hf\middle/k_BT_\mathrm{c}\right.}-1} + \frac{1}{2}\right)\left. hf\middle/k_B\right.$, the blackbody 
radiation from the cavity at a physical temperature $T_\mathrm{c}$ and 
the quantum noise associated with the zero-point fluctuation of the 
vacuum, which are further modulated by a Lorentzian function due to the 
higher temperature at the cavity with respect to that in the 
dilution refrigerator. 
More details may be found in Sec.~\ref{sec:taseh} and 
Appendix~\ref{sec:cavitynoise}. (ii) \ta, the 
noise added by the receiver (mainly from the first-stage amplifier). 
The Lorentzian modulation of $T_\text{cn}$ will be removed from 
the averaged noise spectrum and only the average value of $T_\text{cn}$ 
will be used in the final analysis (Sec.~\ref{sec:ana}).  
%Equation~\eqref{eq:pn} implies 
%that the noise spectrum has little dependence on the frequency 
%(white spectrum) for the narrow bandwidth considered in the experiment. 
%However, apart from the flat baseline as described by 
%Eq.~\eqref{eq:pn}, the noise spectrum observed by TASEH has an additional 
%component with a Lorentzian shape due to the higher temperature at the 
%cavity with respect to the temperature in the dilution refrigerator. 
%The Lorentzian component will be removed from the measured spectrum and only 
%the baseline \tsys\ will be used in the final analysis (Sec.~\ref{sec:ana}). 

Using the operation parameters of TASEH in Table~\ref{tab:tasehbenchmark} and 
the results from the calibration of readout electronics, 
%the values of $T_\text{bb}$, $T_\text{qn}$, and \ta\ are estimated to be 
%about 0.07~K, 0.12~K, and \noise~K, respectively.  
the values of $T_\text{cn}$ (average) and \ta\ are estimated to be about 
0.12~K and \noise~K, respectively.  
Therefore, the value of \tsys\ for TASEH 
is about 2.0--2.3~K, which gives a noise power of approximately 
$\left(1.4-1.6\right)\times 10^{-19}$~W within the 5-kHz axion signal 
line-width, five 
orders of magnitude larger than the signal. Nevertheless, what matters in the 
analysis is the signal significance, or the so-called signal-to-noise ratio 
(SNR) using the standard terminology of axion experiments, i.e. the ratio of 
the signal power to the fluctuation in the averaged noise power spectrum 
$\sigma_n$. 

%Assuming that the amplitude distribution of the noise voltage within a 
%bandwidth is Gaussian, the $\sigma_n$ can be expressed below 
%according to Dicke's Radiometer Equation~\cite{Dicke}: 
According to Dicke's Radiometer Equation~\cite{Dicke}, the $\sigma_n$ 
is given by: 
\begin{eqnarray}
 \sigma_n  &=&  \frac{P_n}{\sqrt{N_\text{avg}}}, \nonumber \\
           &=&  \frac{P_n}{\sqrt{t\Delta f}}, \nonumber \\
           &=&  k_B\tsys\sqrt{\frac{\Delta f}{t}} 
 \label{eq:sigman}
\end{eqnarray}
where $N_\text{avg}$ is the number of noise power spectra used in the 
average; it is related to the data integration time $t$ 
%and the bandwidth over which a single measurement is made $\Delta f$.  
and the resolution bandwidth $\Delta f$.  
Assuming that all the axion signal power falls within $\Delta f$, 
the SNR will therefore be: 
\begin{eqnarray}
   \text{SNR} & = & \frac{P_s}{\sigma_n}, \nonumber \\
              & = & \frac{P_s}{k_B\tsys}\sqrt{\frac{t}{\Delta f}},
 \label{eq:SNR}
\end{eqnarray}  
Combining Eq.~\eqref{eq:ps} and Eq.~\eqref{eq:SNR},
one could see that the SNR is maximized by an experimental setup with 
a strong magnetic field, a large cavity volume, an efficient cavity 
resonant mode, a receiver with low system noise temperature, and a 
long integration time. 










