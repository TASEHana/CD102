%%%%%%%%%%%%%%%%%%%%%%%%%%%%%%%%%%%%%%%%%%%%%%%%%%%%%%%%%%%%%%%%%%%%%%%%%%%%%%%
\section{Introduction} \label{sec:intro}
The axion is a hypothetical particle predicted as a consequence of a  
solution to the strong CP problem~\cite{strongCPI,strongCPII,strongCPIII}, 
i.e. why the product of the charge 
conjugation (C) and parity (P) symmetries is preserved in the strong 
interactions when there is an explicit CP-violating term in the QCD 
Lagrangian. In other words, why is the electric dipole moment 
of the neutron so tiny:  
%$d_n =\left(0.0\pm 1.1_\mathrm{stat} \pm 0.2_\mathrm{sys}\right)\times10^{-26}~e\cdot\mathrm{cm}$~\cite{EDM}? 
$\left|d_n\right| < 1.8 \times10^{-26}~e\cdot\mathrm{cm}$~\cite{EDM,PDG}? 
The solution proposed by Peccei and Quinn is to introduce a new global 
Peccei-Quinn U(1)$_\mathrm{PQ}$ symmetry that is spontaneously broken; the 
axion is the pseudo Nambu-Goldstone boson of 
U(1)$_\mathrm{PQ}$~\cite{strongCPI}. 
Axions are abundantly produced during the QCD phase transition in 
the early universe and may constitute the dark matter (DM). 
In the post-inflationary PQ symmetry breaking scenario, where the PQ symmetry
is broken after inflation, current calculations suggest a mass range of 
1–-100~\muevcc\ for axions so that the cosmic axion density does not exceed 
the 
observed cold DM density~\cite{QCDCalI,QCDCalII,QCDCalIII,QCDCalIV,QCDCalV,QCDCalVI,QCDCalVII,QCDCalVIII,QCDCalIX,QCDCalX,QCDCalXI,QCDCalXII,QCDCalXIII}. 
Refs~\cite{axionDMI,axionDMII,axionDMIII} also suggested that axions form a 
Bose-Einstein condensate; this property explains the occurrence of 
caustic rings in galactic halos. Therefore, axions are compelling because 
they may explain at the same 
time puzzles that are on scales different by more than thirty orders of 
magnitude. 


%
%
%

%However, in the pre-inflationary PQ symmetry breaking scenario, where
%the PQ symmetry is broken before and during inflation and not restored
%afterwards, $m_a$ could have a much smaller value~\cite{PDG}. 
%
%
%

Axions could be detected and studied via their two-photon interaction, the
so-called ``inverse Primakoff effect''. For QCD axions, i.e. the axions 
proposed to solve the strong CP problem, the axion-two-photon coupling 
constant \gagg\ is related to the mass of the axion \ma: 
\begin{equation}
 \gagg = \left(\frac{\ggamma\alpha}{\pi \Lambda^2}\right)\ma,
\end{equation}
where \ggamma\ is a dimensionless model-dependent parameter, $\alpha$ is the 
fine-structure constant, $\Lambda=78~\MeV$ is a scale parameter that can 
be derived from the mass and the decay constant of the pion, and the ratio of 
the up to down quark masses. 
The numerical values of \ggamma\ are -0.97 and 0.36 
in the Kim-Shifman-Vainshtein-Zakharov (KSVZ)~\cite{KSVZI,KSVZII} and 
the Dine-Fischler-Srednicki-Zhitnitsky (DFSZ)~\cite{DFSZI,DFSZII} benchmark 
models, respectively. 
%
%
%Apart from the QCD axions, one could still consider a more general 
%case of axion-like particles (ALPs) where the two parameters \gagg\ 
%and \ma\ are independent from each other 
%and search for ALPs via the two-photon interaction. 
% 

The detectors with the best sensitivities to axions with a mass of 
$\approx \mu\mathrm{eV}/c^2$, as first put forward by 
Sikivie~\cite{SikivieI,SikivieII},  
are haloscopes consisting of a microwave cavity immersed in a strong static 
magnetic field and operated at a cryogenic temperature. 
In the presence of an external magnetic field, the ambient 
oscillating axion field induces an electric current that oscillates with 
a frequency $\nu$ set by the total energy of the axion: 
$h\nu=E_a=\ma c^2 + \frac{1}{2}\ma v^2$. The induced electric current 
and the microwave cavity act as coupled oscillators and resonate when the 
frequencies of the electromagnetic modes in the cavity match $\nu$; the signal 
power is further delivered in the form of microwave photons and 
readout with a low-noise amplifier. The axion mass is unknown, therefore, 
the cavity resonator must allow the possibility to be tuned through a range
of possible axion masses. Over the years, 
the Axion Dark Matter eXperiment (ADMX) had developed and improved the 
cavity design and readout electronics; they excluded KSVZ 
benchmark model within the mass range of %1.9--3.69~$\mu$eV$/c^2$. 
1.9--4.2\muevcc\ and DFSZ benchmark model for the mass ranges 
of 2.66--3.31 and 3.9--4.1\muevcc, 
respectively~\cite{ADMXI,ADMXII,ADMXIII,ADMXIV,ADMXV,ADMXVI,ADMXVII}. 
The Haloscope at Yale Sensitive to Axion Cold dark matter 
(HAYSTAC)~\cite{HAYSTACI} and the 
Center for Axion and Precision Physics Research (CAPP)~\cite{CAPPI} aim for 
axions at higher masses and have pushed the limits on \gagg\ towards the 
KSVZ value for the mass ranges of 16.96--17.12 
and 17.14--17.28\muevcc, and 10.7126--10.7186\muevcc, respectively. This 
paper presents the first results and the analysis details of a search for 
axions for the mass range of \mlo--\mhi\muevcc, from the Taiwan Axion 
Search Experiment with Haloscope (TASEH).  

%%%%%%%%%%%%%%%%%%%%%%%%%%%%%%%%%%%%%%%%%%%%%%%%%%%%%%%%%%%%%%%%%%%
\subsection{The expected axion signal power and signal line shape}
\label{sec:introsignal}

The signal power extracted from a microwave cavity on resonance is given 
by:
\begin{equation}
P_s = \left(\ggamma^2\frac{\alpha^2\hbar^3c^3\rho_a}{\pi^2\Lambda^4}\right)\times
\left(\omega_c\frac{1}{\mu_0}B_0^2VC_{mnl}Q_L\frac{\beta}{1+\beta}\right),
\label{eq:ps}
\end{equation}
where $\rho_a=0.45~\GeV/\mathrm{cm}^3$ is the local dark-matter density. 
The second set of parentheses contains parameters related to the experimental 
setup: the angular resonance frequency of the cavity $\omega_c$, 
the vacuum permeability $\mu_0$, the average strength of the external magnetic 
field $B_0$, the volume of the cavity $V$, and the loaded quality factor of the 
cavity 
\(Q_L=Q_0/(1+\beta)\), where $Q_0$ is the unloaded, intrinsic quality factor 
of the cavity and $\beta$ determines the amount of coupling of the signal to 
the receiver. The form factor $C_{mnl}$ is the normalized 
overlap of the electric field 
$\vec{\bold{E}}$, for a particular cavity resonant mode, with the external magnetic 
field $\vec{\bold{B}}$:
\begin{equation}
  C_{mnl} = \frac{\left[\int\left( \vec{\bold{B}}\cdot\vec{\bold{E}}_{mnl}\right) d^3\bold{x}\right]^2}{B_0^2V\int E_{mnl}^2 d^3\bold{x}}.
\label{eq:formfactor} 
\end{equation} 
Here, the magnetic field $\vec{\bold{B}}$ points mostly along the axial 
direction ($z$-axis) of the cavity. 
The field strength has a small variation along the radial and axial directions and 
$B_0$ is the averaged value over the whole cavity volume. 
For cylindrical cavities, the largest form factor is from the 
TM$_{010}$ mode. The expected signal power derived from the experimental 
parameters of TASEH (see Table~\ref{tab:tasehbenchmark}) 
is $P_s\simeq 1.5\times10^{-24}$~W for a KSVZ axion with a 
mass of 19.5\muevcc. 

In the direct dark matter search experiments, several assumptions were 
made in order to derive a signal line shape. 
The density and the velocity distributions of DM are related to each other 
through the gravitational potential. The DM in the galactic halo is assumed 
to be virialized. The DM halo density distribution is assumed 
to be spherically symmetric and close to be isothermal, which results in a 
velocity distribution similar to the Maxwell-Boltzmann distribution. The 
distribution of the measured signal frequency can be further derived from the 
velocity distribution after a change of variables and set 
\(h\nu_a = \ma c^2\). Previous experimental results typically adopt the 
following function for frequency $\nu\ge\nu_a$: 
\begin{equation}
f(\nu) = \frac{2}{\sqrt{\pi}}\sqrt{\nu-\nu_a}\left(\frac{3}{\alpha}\right)^{3/2}
e^{\frac{-3\left(\nu-\nu_a\right)}{\alpha}}, 
\label{eq:simplesignal}
\end{equation}
where $\alpha\equiv  \nu_a \left<v^2\right>/c^2$ and is related to the 
variance of the velocity distribution. For the Maxwell-Boltzmann distribution, 
$\left<v^2\right>=3v_c^2/2=$(270~km/s)$^2$ where $v_c=220$~km/s is the local 
circular velocity of DM in the galactic rest frame. 
Equation~(\ref{eq:simplesignal}) 
is modified if one considers that the relative velocity of the DM halo with 
respect to the Earth is not the same as the DM velocity in the galactic rest 
frame~\cite{SignalLineShapeI}. The velocity distributions shall also be 
truncated so that the DM velocity is not larger than the escape velocity of 
the Milky Way~\cite{Lisanti:2016jxe}. 
Several N-body simulations~\cite{Diemand:2008in,Springel:2008cc} follow 
structure formation from the initial DM density perturbations to the largest 
halo today and take into account the merger history of the Milky Way, rather 
than assuming that the Milky Way is in a steady state; the simulated results 
 suggest velocity distributions with more high-speed particles relative 
to the Maxwellian case~\cite{Navarro:1995iw,Burkert:1995yz}. However, these 
numerical simulations contain only DM particles; an inclusion of baryons may 
enhance the halo's central density due to a condensation of gas towards the 
center of the halo via an adiabatic 
contraction~\cite{Blumenthal:1985qy,Gnedin:2004cx}, or may reduce 
the density due to the supernova outflows, etc~\cite{Mashchenko:2007jp,Governato:2009bg}. 


In order to compare the results of TASEH with those of the former experiments, 
the analysis presented in this paper assumes an axion 
signal line shape by including Eq.~(\ref{eq:simplesignal}) in the weights 
when merging the measured power from multiple frequency bins 
(see Section~\ref{sec:ana}). Still given the caveats above and a lack of 
strong evidence for any particular choice of velocity distributions, 
the results without an assumption of signal line shape and the results 
with a simple Gaussian weight are also presented for comparison. 
In addition, a signal line width 
$\Delta\nu_a=\ma\left<v^2\right>/h\simeq$~5~kHz, which is much smaller than 
the TASEH cavity line-width $\nu_a/Q_L\simeq$~250~kHz, is assumed and 
five frequency bins are merged to perform the final analysis. For a signal 
line shape as described in Eq.~(\ref{eq:simplesignal}), a 5-kHz bandwidth 
includes about 95\% of the distribution.

%Given the caveats above and a lack of strong evidence for any particular 
%choice of velocity distributions, the central results of the analysis 
%presented in this paper do not make an assumption for the details of the axion 
%signal line shape when merging the measured power from multiple frequency bins,
% i.e. the weights for merging do not include the signal line shape 
%(see Section~\ref{sec:ana}). The results including 
%Eq.~(\ref{eq:simplesignal}) in the weights for merging are presented only 
%for comparison. A signal line width 
%$\Delta\nu_a=\ma\left<v^2\right>/h\simeq$~5~kHz, which is much smaller than 
%the TASEH cavity line-width $\nu_a/Q_L\simeq$~250~kHz, is still assumed and 
%five frequency bins are merged to perform the final analysis. 

 
\subsection{The expected noise and the signal-to-noise ratio}
\label{sec:intronoise}
Several physics processes can contribute to the total noise and all of them 
can be seen as Johnson thermal noise at some effective temperature, or the 
so-called system noise temperature \tsys. The total noise power in a 
bandwidth $\Delta\nu$ is then:
\begin{equation}
  P_n = k_B\tsys\Delta\nu, 
\end{equation}
where $k_B$ is the Boltzmann constant. 
The system noise temperature \tsys\ has three major components: 
\begin{equation}
  k_B\tsys = h\nu\left(\frac{1}{e^{\left.h\nu\right/k_{B}T_\mathrm{cavity}}-1} + \frac{1}{2} \right) + k_B\ta. 
\label{eq:pn}
\end{equation}
 The three terms in Eq.~(\ref{eq:pn}) correspond to: (i) the blackbody radiation 
from the cavity at temperature $T_\mathrm{cavity}$, (ii) the quantum noise 
associated with the zero-point fluctuation of the blackbody gas, and (iii) the 
noise added by the receiver, which is expressed in terms of an effective 
temperature \ta. The first term in Eq.~(\ref{eq:pn}) implies 
that the noise spectrum from the cavity has little dependence on the frequency 
(white spectrum) for the narrow bandwidth considered in the experiment. 
However, the noise spectrum observed by TASEH 
was actually Lorentzian due to a temperature difference between the cavity 
and the transmission line in the dilution refrigerator. More details may be 
found in Section~\ref{sec:taseh} and Appendix~\ref{sec:cavitynoise}. 

Using the operation parameters of TASEH in Table~\ref{tab:tasehbenchmark} and 
the results from the calibration of readout electronics, 
the effective temperatures of these three sources are estimated to be about 
0.07~K, 0.12~K, and \noise~K, respectively.  
Therefore, the value of \tsys\ for TASEH 
is about 2.1--2.4~K, which gives a noise power of approximately 
$\left(1.5-1.7\right)\times 10^{-19}$~W 
for a bandwidth of 5~kHz (the assumed axion signal line-width), three 
orders of magnitude larger than the signal. Nevertheless, what matters in the 
analysis is the signal significance, or the so-called signal-to-noise ratio 
(SNR) using the standard terminology of axion experiments, i.e. the ratio of 
the signal power to the uncertainty in the estimation of 
the noise power:
\begin{eqnarray}
   \text{SNR} & = & \frac{P_s}{\delta P_n} = \frac{P_s}{P_n}\sqrt{\Delta\nu_a\tau}, \nonumber \\
              & = & \frac{P_s}{k_B\tsys}\sqrt{\frac{\tau}{\Delta\nu_a}},
 \label{eq:SNR}
\end{eqnarray}  
where $\tau$ is the amount of data integration time. Equation~(\ref{eq:SNR}) 
can be derived from Dicke's Radiometer Equation, assuming that the amplitude 
distribution of the noise voltage within a bandwidth $\Delta\nu_a$ is Gaussian.










