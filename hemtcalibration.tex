\section{The calibration}
\label{sec:hemtcalibration}

The first stage of our amplication chain is HEMT whose adding noise ($T_A$) mainly contributes to the noise of the system. The noise is one of the most important parameters for the axion search. Therefore, calibration for the HEMT is crucial part for our experiment. To calibrate, HEMT was connected to the heat source instead of cavity, several input currents were delivered to the source to change its temperature monitored by a thermometer. The powers from the source went through HEMT and the amplifier outside the DR to the signal analyzer as in axion data running. The received powers were linearly fitted as a function of the source temperature to extract the gain and adding noise for the HEMT, more details of the process can be found in Ref. \cite{}. The calibration was carried out before, during and after the data taking which showed that the HEMT performance is stable over time. The average adding noise from HEMT $T_A$ has lowest value of 1.9K at frequency of 4.8 GHz and highest value of 2.2K at 4.72 GHz, as presented in Fig. \cite{}. The error bars are the RMS of $T_A$ and the largest RMS was used to calculate the systematic uncertainty for \gagg limit due to the noise. The light blue points in Fig. \cite{} are the noise from the axion data estimated by removing gain and subtracting the cavity temperature. It is shown good agreement between calibrated and estimated results. The biggest difference is 0.076K in the frequency range at which the data were recorded after an earthquake. We do not know what source of the problem, therefore, the difference was quoted as systematic uncertainy together with the RMS noise.

\begin{figure} [h]
  \centering
  \includegraphics[width=0.4\textwidth,height = 0.25\textwidth]{Figure/}
  \caption{The calibrated average adding noise of the HEMT in frequency}
\end{figure}


  

