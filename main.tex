% ****** Start of file apssamp.tex ******
%
%   This file is part of the APS files in the REVTeX 4.2 distribution.
%   Version 4.2a of REVTeX, December 2014
%
%   Copyright (c) 2014 The American Physical Society.
%
%   See the REVTeX 4 README file for restrictions and more information.
%
% TeX'ing this file requires that you have AMS-LaTeX 2.0 installed
% as well as the rest of the prerequisites for REVTeX 4.2
%
% See the REVTeX 4 README file
% It also requires running BibTeX. The commands are as follows:
%
%  1)  latex apssamp.tex
%  2)  bibtex apssamp
%  3)  latex apssamp.tex
%  4)  latex apssamp.tex
%
\documentclass[%
% reprint, %% for final paper
%superscriptaddress,
%groupedaddress,
%unsortedaddress,
%runinaddress,
%frontmatterverbose, 
preprint, %% for single-column, double-spacing
%preprintnumbers,
%nofootinbib,
%nobibnotes,
%bibnotes,
 amsmath,amssymb,
 aps,
%pra,
%prb,
%rmp,
%prstab,
%prstper,
%floatfix,
]{revtex4-2}
\usepackage[utf8]{inputenc}
\usepackage{graphicx}% Include figure files
\usepackage{dcolumn}% Align table columns on decimal point
\usepackage{bm}% bold math
\usepackage{hyperref}% add hypertext capabilities
\usepackage[mathlines]{lineno}% Enable numbering of text and display math
\linenumbers\relax % Commence numbering lines

%\usepackage[showframe,%Uncomment any one of the following lines to test 
%%scale=0.7, marginratio={1:1, 2:3}, ignoreall,% default settings
%%text={7in,10in},centering,
%%margin=1.5in,
%%total={6.5in,8.75in}, top=1.2in, left=0.9in, includefoot,
%%height=10in,a5paper,hmargin={3cm,0.8in},
%]{geometry}

%abbreviation
\newcommand{\gagg}{\ensuremath{g_{a\gamma\gamma}}}
\newcommand{\ggamma}{\ensuremath{g_{\gamma}}}
\newcommand{\ma}{\ensuremath{m_a}}
%\def\muev{\ensuremath{\mu~\mathrm{eV}}
\newcommand{\tsys}{\ensuremath{\text{T}_\text{sys}}}
\newcommand{\muevcc}{\ensuremath{\,\mu\text{e\hspace{-.08em}V\hspace{-0.16em}/\hspace{-0.08em}}c^2}}
\newcommand{\MeV}{\ensuremath{\,\text{Me\hspace{-.08em}V}}}
\newcommand{\GeV}{\ensuremath{\,\text{Ge\hspace{-.08em}V}}}
\newcommand{\GeVinv}{\ensuremath{\,\text{Ge\hspace{-.08em}V}^{-1}}}
\newcommand{\flo}{\ensuremath{4.707506}}
\newcommand{\fhi}{\ensuremath{4.798145}}
\newcommand{\mlo}{\ensuremath{xxxx}}
\newcommand{\mhi}{\ensuremath{yyyy}}
\newcommand{\avelimit}{\ensuremath{zzzz\times 10^{-13}}}
%\newcommand{\noise}{\ensuremath{2.3}}
\newcommand{\noise}{\ensuremath{1.9 - 2.2}}


\begin{document}

\preprint{APS/123-QED}

\title{First Results from the Taiwan Axion Search Experiment with Haloscope at 20\muevcc}% Force line breaks with \\
\thanks{A footnote to the article title}%

\author{Ann Author}
 \altaffiliation[Also at ]{Physics Department, XYZ University.}%Lines break automatically or can be forced with \\
\author{Second Author}%
 \email{Second.Author@institution.edu}
\affiliation{%
 Authors' institution and/or address\\
 This line break forced with \textbackslash\textbackslash
}%

\collaboration{TASEH Collaboration}%\noaffiliation


\date{\today}% It is always \today, today,
             %  but any date may be explicitly specified

\begin{abstract}

 This paper presents the first results from the 
Taiwan Axion Search Experiment with Haloscope, a search for axions 
using a microwave cavity at frequencies between \flo\ and \fhi~GHz. 
Apart from external signals, no candidates with significance more than 
3.355~$\sigma$ were found. The experiment excludes models with the axion-two-photon 
coupling $\gagg\gtrsim \avelimit\GeVinv$, a factor of ten above the benchmark 
KSVZ model for the mass range $\mlo < \ma < \mhi \muevcc$. For the first time, 
constraints on the $\gagg$ have been placed in this mass region. 


%\begin{description}
%\item[Usage]
%Secondary publications and information retrieval purposes.
%\item[Structure]
%You may use the \texttt{description} environment to structure your abstract;
%use the optional argument of the \verb+\item+ command to give the category of each item. 
%\end{description}
\end{abstract}

%\keywords{Suggested keywords}%Use showkeys class option if keyword
                              %display desired
\maketitle

\tableofcontents
%%%%%%%%%%%%%%%%%%%%%%%%%%%%%%%%%%%%%%%%%%%%%%%%%%%%%%%%%%%%%%%%%%%%%%%%%%%%%%%
\section{Introduction} \label{sec:intro}
The axion is a hypothetical particle predicted as a consequence of a  
solution to the strong CP problem~\cite{strongCPI,strongCPII,strongCPIII}, 
i.e. why the product of the charge 
conjugation (C) and parity (P) symmetries is preserved in the strong 
interactions when there is an explicit CP-violating term in the QCD 
Lagrangian. In other words, why is the electric dipole moment 
of the neutron so tiny:  
%$d_n =\left(0.0\pm 1.1_\mathrm{stat} \pm 0.2_\mathrm{sys}\right)\times10^{-26}~e\cdot\mathrm{cm}$~\cite{EDM}? 
$\left|d_n\right| < 1.8 \times10^{-26}~e\cdot\mathrm{cm}$~\cite{EDM,PDG}? 
The solution proposed by Peccei and Quinn is to introduce a new global 
Peccei-Quinn U(1)$_\mathrm{PQ}$ symmetry that is spontaneously broken; the 
axion is the pseudo Nambu-Goldstone boson of 
U(1)$_\mathrm{PQ}$~\cite{strongCPI}. 
Axions are abundantly produced during the QCD phase transition in 
the early universe and may constitute the dark matter (DM). 
In the post-inflationary PQ symmetry breaking scenario, where the PQ symmetry
is broken after inflation, current calculations suggest a mass range of 
1–-100~\muevcc\ for axions so that the cosmic axion density does not exceed 
the 
observed cold DM density~\cite{QCDCalI,QCDCalII,QCDCalIII,QCDCalIV,QCDCalV,QCDCalVI,QCDCalVII,QCDCalVIII,QCDCalIX,QCDCalX,QCDCalXI,QCDCalXII,QCDCalXIII}. 
Refs~\cite{axionDMI,axionDMII,axionDMIII} also suggested that axions form a 
Bose-Einstein condensate; this property explains the occurrence of 
caustic rings in galactic halos. Therefore, axions are compelling because 
they may explain at the same 
time puzzles that are on scales different by more than thirty orders of 
magnitude. 


%
%
%

%However, in the pre-inflationary PQ symmetry breaking scenario, where
%the PQ symmetry is broken before and during inflation and not restored
%afterwards, $m_a$ could have a much smaller value~\cite{PDG}. 
%
%
%

Axions could be detected and studied via their two-photon interaction, the
so-called ``inverse Primakoff effect''. For QCD axions, i.e. the axions 
proposed to solve the strong CP problem, the axion-two-photon coupling 
constant \gagg\ is related to the mass of the axion \ma: 
\begin{equation}
 \gagg = \left(\frac{\ggamma\alpha}{\pi \Lambda^2}\right)\ma,
\end{equation}
where \ggamma\ is a dimensionless model-dependent parameter, $\alpha$ is the 
fine-structure constant, $\Lambda=78~\MeV$ is a scale parameter that can 
be derived from the mass and the decay constant of the pion, and the ratio of 
the up to down quark masses. 
The numerical values of \ggamma\ are -0.97 and 0.36 
in the Kim-Shifman-Vainshtein-Zakharov (KSVZ)~\cite{KSVZI,KSVZII} and 
the Dine-Fischler-Srednicki-Zhitnitsky (DFSZ)~\cite{DFSZI,DFSZII} benchmark 
models, respectively. 
%
%
%Apart from the QCD axions, one could still consider a more general 
%case of axion-like particles (ALPs) where the two parameters \gagg\ 
%and \ma\ are independent from each other 
%and search for ALPs via the two-photon interaction. 
% 

The detectors with the best sensitivities to axions with a mass of 
$\approx \mu\mathrm{eV}/c^2$, as first put forward by 
Sikivie~\cite{SikivieI,SikivieII},  
are haloscopes consisting of a microwave cavity immersed in a strong static 
magnetic field and operated at a temperature 
below 0.1~K. In the presence of an external magnetic field, the ambient 
oscillating axion field induces an electric current that oscillates with 
a frequency $\nu$ set by the total energy of the axion: 
$h\nu=E_a=\ma c^2 + \frac{1}{2}\ma v^2$. The induced electric current 
and the microwave cavity act as coupled oscillators and resonate when the 
frequencies of the electromagnetic modes in the cavity match $\nu$; the signal 
power is further delivered in the form of microwave photons and 
readout with a low-noise amplifier. The axion mass is unknown, therefore, 
the cavity resonator must allow the possibility to be tuned through a range
of possible axion masses. Over the years, 
the Axion Dark Matter eXperiment (ADMX) had developed and improved the 
cavity design and readout electronics and excluded KSVZ 
benchmark model within the mass range of %1.9--3.69~$\mu$eV$/c^2$. 
1.9--4.2\muevcc\ and DFSZ benchmark model for the mass range 
of 2.66--3.31\muevcc~\cite{ADMXI,ADMXII,ADMXIII,ADMXIV,ADMXV,ADMXVI,ADMXVII}. 
The Haloscope at Yale Sensitive to Axion Cold dark matter 
(HAYSTAC)~\cite{HAYSTACI} and the 
Center for Axion and Precision Physics Research (CAPP)~\cite{CAPPI} aim for 
axions at higher masses and have pushed the limits on \gagg\ towards the 
KSVZ value for the mass ranges of 16.96--17.12 
and 17.14--17.28\muevcc, and 10.7126--10.7186\muevcc, respectively. This 
paper presents the first results and the analysis details of a search for 
axions for the mass range of \mlo--\mhi\muevcc, from the Taiwan Axion 
Search Experiment with Haloscope (TASEH).  

%%%%%%%%%%%%%%%%%%%%%%%%%%%%%%%%%%%%%%%%%%%%%%%%%%%%%%%%%%%%%%%%%%%
\subsection{The expected axion signal power and signal line shape}

The signal power extracted from a microwave cavity on resonance is given 
by:
\begin{equation}
P_s = \left(\ggamma^2\frac{\alpha^2\hbar^3c^3\rho_a}{\pi^2\Lambda^4}\right)\times
\left(\omega_c\frac{1}{\mu_0}B_0^2VC_{mnl}Q_L\frac{\beta}{1+\beta}\right),
\end{equation}
where $\rho_a=0.45~\GeV/\mathrm{cm}^3$ is the local dark-matter density. 
The second set of parentheses contains parameters related to the experimental 
setup: the angular resonance frequency of the cavity $\omega_c$, 
the vacuum permeability $\mu_0$, the average strength of the external magnetic 
field $B_0$, the volume of the cavity $V$, and the loaded quality factor of the 
cavity 
\(Q_L=Q_0/(1+\beta)\), where $Q_0$ is the unloaded, intrinsic quality factor 
of the cavity and $\beta$ determines the amount of coupling of the signal to 
the receiver. The form factor $C_{mnl}$ is the normalized 
overlap of the electric field 
$\vec{\bold{E}}$, for a particular cavity resonant mode, with the external magnetic 
field $\vec{\bold{B}}$:
\begin{equation}
  C_{mnl} = \frac{\left[\int\left( \vec{\bold{B}}\cdot\vec{\bold{E}}_{mnl}\right) d^3\bold{x}\right]^2}{B_0^2V\int E_{mnl}^2 d^3\bold{x}}.
\label{eq:formfactor} 
\end{equation} 
Here, the magnetic field $\vec{\bold{B}}$ points mostly along the axial 
direction ($z$-axis) of the cavity. 
The field strength has a small variation along the radial and axial directions and 
$B_0$ is the averaged value over the whole cavity volume. 
For cylindrical cavities, the largest form factor is from the 
TM$_{010}$ mode. The expected signal power derived from the experimental 
parameters of TASEH (see Table~\ref{tab:tasehbenchmark}) 
is $P_s\simeq 1.5\times10^{-24}$~W for a KSVZ axion with a 
mass of 19.5\muevcc. 

In most of the direct dark matter search experiments, several assumptions were 
made in order to derive a signal line shape. 
The density and the velocity distributions of DM are related to each other 
through the gravitational potential. The DM in the galactic halo is assumed 
to be virialized. The DM halo density distribution is assumed 
to be spherically symmetric and close to be isothermal, which results in a 
velocity distribution similar to the Maxwell-Boltzmann distribution. The 
distribution of the measured signal frequency can be further derived from the 
velocity distribution after a change of variables and set 
\(h\nu_a = \ma c^2\). Previous experimental results typically adopt the 
following function for frequency $\nu\ge\nu_a$: 
\begin{equation}
f(\nu) = \frac{2}{\sqrt{\pi}}\sqrt{\nu-\nu_a}\left(\frac{3}{\alpha}\right)^{3/2}
e^{\frac{-3\left(\nu-\nu_a\right)}{\alpha}}, 
\label{eq:simplesignal}
\end{equation}
where $\alpha\equiv  \nu_a \left<v^2\right>/c^2$ and is related to the 
variance of the velocity distribution. For the Maxwell-Boltzmann distribution, 
$\left<v^2\right>=3v_c^2/2=$(270~km/s)$^2$ where $v_c=220$~km/s is the local 
circular velocity of DM in the galactic rest frame. 
Equation~(\ref{eq:simplesignal}) 
is modified if one considers that the relative velocity of the DM halo with 
respect to the Earth is not the same as the DM velocity in the galactic rest 
frame~\cite{SignalLineShapeI}. The velocity distributions shall also be 
truncated so that the DM velocity is not larger than the escape velocity of 
the Milky Way~\cite{Lisanti:2016jxe}. 
Several N-body simulations~\cite{Diemand:2008in,Springel:2008cc} follow 
structure formation from the initial DM density perturbations to the largest 
halo today and take into account the merger history of the Milky Way, rather 
than assuming that the Milky Way is in a steady state; the simulated results 
 suggest velocity distributions with more high-speed particles relative 
to the Maxwellian case~\cite{Navarro:1995iw,Burkert:1995yz}. However, these 
numerical simulations contain only DM particles; an inclusion of baryons may 
enhance the halo's central density due to a condensation of gas towards the 
center of the halo via an adiabatic 
contraction~\cite{Blumenthal:1985qy,Gnedin:2004cx}, or may reduce 
the density due to the supernova outflows, etc~\cite{Mashchenko:2007jp,Governato:2009bg}. 


In order to compare the results of TASEH with those of the former experiments, 
the analysis presented in this paper assumes an axion 
signal line shape by including Eq.~(\ref{eq:simplesignal}) in the weights 
when merging the measured power from multiple frequency bins 
(see Section~\ref{sec:ana}). Nevertheless, given the caveats above and a lack of 
strong evidence for any particular choice of velocity distributions, 
the results without an assumption of signal line shape and the results 
with a simple Gaussian weight are also presented for comparison. 
In addition, a signal line width 
$\Delta\nu_a=\ma\left<v^2\right>/h\simeq$~5~kHz, which is much smaller than 
the TASEH cavity line-width $\nu_a/Q_L\simeq$~250~kHz, is assumed and 
five frequency bins are merged to perform the final analysis. 

%Given the caveats above and a lack of strong evidence for any particular 
%choice of velocity distributions, the central results of the analysis 
%presented in this paper do not make an assumption for the details of the axion 
%signal line shape when merging the measured power from multiple frequency bins,
% i.e. the weights for merging do not include the signal line shape 
%(see Section~\ref{sec:ana}). The results including 
%Eq.~(\ref{eq:simplesignal}) in the weights for merging are presented only 
%for comparison. A signal line width 
%$\Delta\nu_a=\ma\left<v^2\right>/h\simeq$~5~kHz, which is much smaller than 
%the TASEH cavity line-width $\nu_a/Q_L\simeq$~250~kHz, is still assumed and 
%five frequency bins are merged to perform the final analysis. 

 
\subsection{The expected noise and the signal-to-noise ratio}
Several physics processes can contribute to the total noise and all of them 
can be seen as Johnson thermal noise at some effective temperature, or the 
so-called system noise temperature \tsys. The total noise power in a 
bandwidth $\Delta\nu$ is then:
\begin{equation}
  P_n = k_B\tsys\Delta\nu, 
\end{equation}
where $k_B$ is the Boltzmann constant. 
The system noise temperature \tsys\ has three major components: 
\begin{equation}
  k_B\tsys = h\nu\left(\frac{1}{e^{\left.h\nu\right/k_{B}T_\mathrm{cavity}}-1} + \frac{1}{2} + N_A\right). 
\label{eq:pn}
\end{equation}
 The three terms in Eq.~(\ref{eq:pn}) correspond to the blackbody radiation 
from the cavity at temperature $T_\mathrm{cavity}$, the quantum noise 
associated with the zero-point fluctuation of the blackbody gas, and the noise 
added by the receiver $N_A$. The first term in Eq.~(\ref{eq:pn}) implies 
that the noise spectrum from the cavity has little dependence on the frequency 
(white spectrum) for the narrow bandwidth considered in the experiment. 
However, the noise spectrum observed by TASEH 
was actually Lorentzian due to the temperature difference between the cavity 
and the transmission line in the dilution refrigerator. More details may be 
found in Section~\ref{sec:taseh} and Appendix~\ref{sec:cavitynoise}. 

Using the operation parameters of TASEH in Table~\ref{tab:tasehbenchmark}, 
the effective temperatures of these three sources are estimated to be about 
0.07~K, 0.11~K, and 2.1~K, respectively.  
Therefore, the value of \tsys\ for TASEH 
is about \noise~K, which gives a noise power of approximately $1.6\times 10^{-19}$~W 
for a bandwidth of 5~kHz (the assumed axion signal line-width), three 
orders of magnitude larger than the signal. Nevertheless, what matters in the 
analysis is the signal significance, or the so-called signal-to-noise ratio 
(SNR) using the standard terminology of axion experiments, i.e. the ratio of 
the signal power to the uncertainty in the estimate of 
the noise power:
\begin{eqnarray}
   \text{SNR} & = & \frac{P_s}{\delta P_n} = \frac{P_s}{P_n}\sqrt{\Delta\nu_a\tau}, \nonumber \\
              & = & \frac{P_s}{k_B\tsys}\sqrt{\frac{\tau}{\Delta\nu_a}},
 \label{eq:SNR}
\end{eqnarray}  
where $\tau$ is the amount of data integration time. Equation~(\ref{eq:SNR}) 
can be derived from Dicke's Radiometer Equation, assuming that the amplitude 
distribution of the noise voltage within a bandwidth $\Delta\nu_a$ is Gaussian.












\section{Experimental Setup}\label{sec:taseh} 
The detector of TASEH is located at the Department of Physics, National 
Central University, Taiwan and housed within a cryogen-free dilution 
refrigerator (DR) from BlueFors. An 8-Tesla superconducting solenoid 
with a 
bore diameter of 76~mm and a length of 240 mm is integrated with the DR. 

The data for the analysis presented in this paper were collected by TASEH 
from October 13, 2021 to November 15, 2021, and termed as the CD102 data, 
where CD stands for ``cool down''. 
During the data taking, the cavity sat in the center of the magnet bore 
and was connected via holders to the mixing flange of the DR at a 
temperature of $T_{\rm mx}\simeq27$~mK. 
The temperature of the cavity stayed at $T_\text{c}\simeq155$~mK, higher 
with respect to the 
DR; it is believed that the cavity had an unexpected thermal contact with the 
radiation shield in the DR. 
The cavity, made of oxygen-free high-conductivity (OFHC) copper, has an 
effective volume of 0.234~L and is a two-cell cylinder split along 
the axial direction. 
The cylindrical cavity has an inner radius of 2.5~cm and a 
height of 12~cm.  In order to maintain a smooth surface, the cavity underwent 
the processes of annealing, polishing, and chemical cleaning. The resonant 
frequency of the TM$_{010}$ mode at the cryogenic temperature 
can be tuned over the range of 
%4.717--4.999~GHz via the rotation of an off-axis OFHC copper tuning rod, from 
4.667--4.959~GHz via the rotation of an off-axis OFHC copper tuning rod, from 
the position closer to the cavity wall to the position closer to the cavity 
center (i.e. when the vector from the rotation axis to the tuning rod is 
at an angle of $0^\circ$ to $180^\circ$, with respect to the vector from the 
cavity center to the rotation axis). 
%
Over the frequency range of the CD102 run, the form factor $C_{010}$ as 
defined in Eq.~\eqref{eq:formfactor} varies from 0.60 to 0.61 and 
%0.598 to 0.613 and  
the intrinsic, unloaded quality factor $Q_0$ at the cryogenic temperature 
($T_\mathrm{c}\simeq 155$~mK) is $\simeq 60700$. 
The values of $C_{010}$ are derived from the magnetic field map provided by 
BlueFors and the cavity electric field distribution simulated with 
 Ansys HFSS (high-frequency structure simulator).  

An output probe, made of a 50-$\Omega$ semi-rigid coaxial cable that was 
soldered to an SMA (SubMiniature version A) connector, was inserted into the 
cavity and its depth was set for 
$\beta\simeq2$; the optimization of the value of $\beta$
is discussed in more detail in Ref.~\cite{TASEHInstrumentation}. 
  The signal from the output probe was directed to an 
impedance-matched amplification chain. The first-stage amplifier was 
a low noise high-electron-mobility transistor (HEMT) amplifier with an 
effective noise temperature of $\approx 2$~K, mounted on the 4K flange. 
The signal was further amplified at room temperature via a 
three-stage post-amplifier, and down-converted 
and demodulated to in-phase (I) and quadrature (Q) components and digitized 
by an analog-to-digital converter with a sampling rate of 2~MHz. 
%The frequency resolution of the spectra was 1~kHz.

The CD102 data cover the frequency range of \flo--\fhi~GHz. In this paper, most
 of the frequencies in unit of GHz are quoted with five decimal places as the 
absolute accuracy of frequency is $\approx 10$~kHz. It shall be noted that the 
frequency resolution is 1~kHz.  
There were 837 resonant-frequency steps in total, with a frequency difference 
of $\Delta f_\text{s}=95-115$~kHz between the steps. 
The value of $\Delta f_\text{s}$ was kept within 10\% of 
the nominal value 105~kHz ($\lesssim$ half of the cavity 
line width), rather than 
a fixed value, such that the rotation angle of the tuning rod did not need to 
be fine-tuned and the operation time could be minimized. A 10\% variation of 
the $\Delta f_\text{s}$ is found to have no impact on the \gagg\ limits. 
Each resonant-frequency step is denoted as a ``scan'' 
and the data integration time was about 32-42 minutes. The integration 
time was determined based on the target \gagg\ limits and the experimental 
parameters in Table~\ref{tab:tasehbenchmark}; the variation of the integration 
time aimed to remove the frequency-dependence in the \gagg\ limits caused by   
frequency dependence of the added noise \ta. 

A more detailed description of the TASEH detector, the operation of the 
data run, and the calibration of the gain and added noise temperature of the 
whole amplification chain can be found in Ref.~\cite{TASEHInstrumentation}. 
See Table~\ref{tab:tasehbenchmark} for the benchmark experimental parameters 
that can be used to estimate the sensitivity of TASEH. 

\begin{table}
\caption{The benchmark experimental parameters for estimating the sensitivity 
of TASEH. The definitions of the parameters can be found in Sec.~\ref{sec:intro}. 
More details regarding the determination and the measurements of 
some of the parameters may be found in Ref.~\cite{TASEHInstrumentation}.} 
\label{tab:tasehbenchmark}
\begin{center}
\begin{tabular}{cr}
\hline\hline
 $f_\mathrm{lo}$ & \flo~GHz\\
 $f_\mathrm{hi}$ & \fhi~GHz \\
% $N_\text{step}$ & 839 \\
 $N_\text{step}$ & 837 \\
 $\Delta f_\text{s}$ & 95 -- 115 kHz \\
% $f_\text{resolution}$ & 1 kHz \\
 $B_0$  & 8 Tesla \\
 $V$ & 0.234 L \\ %234255~mm$^3$ 
% $C_{010}$ & 0.598 -- 0.613 \\
 $C_{010}$ & 0.60 -- 0.61 \\
 $Q_0$ & 58000 -- 65000 \\
 $\beta$ & 1.9 -- 2.3 \\
 $T_{\rm mx}$ & 27--28~mK\\
 $T_\mathrm{c}$ & 155~mK \\
 \ta & \noise~K \\
 $\Delta f_a$ & 5 kHz \\
\hline\hline
\end{tabular}
\end{center}
\end{table}

\section{Analysis Procedure} \label{sec:ana}
% \begin{flushleft}
%    \subsection{Analysis overview}
The goal of TASEH is to find the axion signal hidden in the noise. In 
order to achieve this, the analysis procedure includes the following steps:
    \begin{enumerate}
        %\item Read raw data (I, Q) from tdms file and do Fast Fourier transform every 1 millisecond of spectrum, then average all over spectra in 1 second to get power spectrum.
        \item Perform fast Fourier transform (FFT) on the 
time-dependent spectrum to obtain the frequency-dependent spectrum.
        %\item Divide the measured power by the gains from the receiver chain to retrieve real power from cavity.
        %\item HEMT drift to predict noise and gain from the environmental parameters.
        %\item Use the results of high-electron-mobility-transistor(HEMT) drifting (see Section \ref{hemt_drifting}) to predict adding noise and gain; the input environmental parameters include monitoring temperatures, voltages of power supplies, etc.
        %\item Use Savitzky-Golay filter \cite{SGFilter} to remove the spectrum structure, then subtract the mean power.
        \item Apply the Savitzky-Golay (SG) filter to remove the structure 
of the background in the frequency-dependent spectrum.
        \item Combine all power spectra from different frequency scans with 
the weighting algorithm.
        %\item Cause the expected Axion bandwidth, and our frequency resolution is 1KHz, so we need to merge the bin.
        \item Merge bins in the combined spectrum to maximize the SNR. 
%        %\item Set threshold at ${3.355\ \sigma}$.
       \item Rescan the frequency regions with candidates and set limits on 
      the axion-two-photon coupling \gagg\ if no candidates were found.
    \end{enumerate}

    The analysis is done by following the procedure similar to that 
adopted by the HAYSTAC experiment ~\cite{HAYSTACII}.
% \end{flushleft}

%\subsection{Tdms}
\subsection{Fast Fourier transform}
The in-phase $I(t)$ and quadrature $Q(t)$ components of the time-dependent 
data were recorded and saved in the TDMS 
(Technical Data Management Streaming) files - a 
binary format developed by National Instruments.
%Fast Fourier Transform (FFT) was performed to convert the data into frequency-dependent and into unit of power by using the equation:
The fast Fourier Transform (FFT) is performed to convert the data into 
frequency-dependent spectrum in which the measured power is calculated 
using the following equation:

\begin{equation}
\label{eq:4.1}
    \text{Power} = \frac{|\text{FFT}(I+i \cdot Q)|^{2}}{N \cdot 2R},
\end{equation}
where $N$ is the number of data points ($N  = 2000$ in the TASEH 
CD102 data), and $R$ is the resistance of the signal analyzer (50~$\Omega$).
The FFT is done for every one-millisecond subspectrum data. The integration 
time for each frequency scan was about 32-42 minutes, which resulted 
in 1920000 to 2520000 subspectra; an average over these subspectra gives 
the averaged frequency-dependent spectrum for each scan. 
The frequency span in the spectrum from each resonant-frequency scan is 
1.6~MHz while the 
resolution is 1~kHz, giving 1600 frequency bins in each spectrum.  

%\subsection{Savitzky Golay filter}
\subsection{Remove the structure of the background}
%Figure \ref{fig:sg_result} is the spectrum in each step, somehow they all have a similar structure, in order to remove this structure, we will use a Savitzky Golay filter in every second to smooth the data Figure \ref{fig:After_Sg}.

In the absence of the axion signal, the output data spectrum is simply the 
noise from the cavity and the amplification chain. If axions are present 
in the cavity, the signal will be buried in the noise because the 
signal power is very weak. Therefore, the structure of the raw 
output power spectrum, as shown in Fig.~\ref{fig:raw_sg_power}, is dominated 
by the noise of the system and an explanation for the structure can be found 
in Appendix~\ref{sec:cavitynoise}. The Savitzky Golay (SG) 
filter~\cite{SGFilter}, a digital filter that can smooth data without 
distorting the signal tendency, is applied to remove the structure of the  
background. The SG filter is performed on the averaged spectrum of each 
frequency scan by fitting adjacent points of successive sub-sets of data with 
an $n^\text{th}$-order polynomial. The result depends on two parameters: 
the number of 
data points used for fitting, the so-called window width, and the order of 
the polynomial. If the window is too wide, the filter will not remove small 
structures, and if it is too narrow, it may kill the signal. 
The window and the order were first chosen during the data taking based on 
the structure of data and the ratio of the raw data to the filter output. 
After the data taking, they were optimized by injecting an axion signal on 
top of 
the noise data and found that they were consistent with the original choice 
(see Sec.~\ref{sec:sys}). 

The raw averaged spectrum is divided by the output of the SG filter, then  
unity is subtracted from the ratio to get the normalized spectrum 
(Fig.~\ref{fig:raw_sg_power}).
Therefore, if the axion signal exists, a power excess will be above zero.
%After filtering, the normalized spectrum (Fig. \cite{fig:}) was obtained by dividing the raw spectrum by the output of the SG filter subtract 1. Therefore, if the signal exists, the excess power will be above 0.
During the data taking, the resonant frequency of the cavity was  
adjusted by the tuning bar so to scan a large range of frequencies and to 
reduce the uncertainty of the noise at the overlapped region. Therefore, the 
spectra of all the scans need to be combined to create one big spectrum. 
Before doing this, 
the normalized spectrum from each scan is rescaled by the system noise 
(detailed in Sec~\ref{sec:intronoise} and Sec~\ref{sec:calibration}) and the 
signal power with the Lorentzian cavity response taken into account. 
The system-noise temperature \tsys\ is calculated following Eq.~\eqref{eq:pn},
 where the frequency dependence of the added-noise temperature \ta\ is 
obtained from the fitting function in Fig.~\ref{fig:hemtcalvsf}. The rescaled 
spectrum, shown in 
Fig.~\ref{fig:rescaled_power}, is computed with the following formula:

\begin{equation}
  \label{eq:respower_eqn}
  \delta_{ij}^\text{res} = \frac{k_{B}\tsys \Delta\nu }{P_{ij}^{s} h}\delta_{ij}^\text{norm},
\end{equation}

and the standard deviation of each bin is:
\begin{equation}
  \label{eq:ressigma_eqn}
  \sigma_{ij}^\text{res} = \frac{k_{B}\tsys\Delta\nu }{P_{ij}^{s} h}\sigma_{i}^\text{norm},
\end{equation}
where $\delta_{ij}^\text{norm}$ ($\delta_{ij}^\text{res}$) and 
$\sigma_{i}^\text{norm}$ ($\sigma_{ij}^\text{res}$) are the 
normalized (rescaled) power and the 
standard deviation of the $j^\text{th}$ frequency bin from the 
$i^\text{th}$ resonant-frequency scan. 
The $\Delta\nu$ is the bin 
width of spectrum (1~kHz) and  $P_{ij}^{s}$ is the KSVZ axion signal power. 
The $h = \frac{1}{1 + [2(\nu_{ij} - \nu_{ci})/\Delta\nu_{i}]^2}$ 
describes the Lorentzian response of the cavity, where 
$\Delta\nu_{i}$ is the cavity line width and depends on the resonant 
frequency $\nu_{ci}$ and the loaded quality factor. 
%
If a signal appears in a certain frequency bin $j$, its expected power 
will vary depending on the bin position due to the cavity's 
Lorentzian response. The rescaling will take into account this effect. 
The procedure of the normalization and the rescaling also ensures that a 
KSVZ axion signal will have a power $\delta_{ij}^\text{res}$ 
that is approximately equal to unity. 

\begin{figure} [htbp]
  \centering
  \includegraphics[width=8.6cm]{figures/RawPower_SGPower_Ratio_vs_Freq_Step_0100.png}
  \caption{Upper panel: The raw power spectrum (red points) and the output of 
the SG filter (blue curve) of one scan. Bottom panel: The normalized 
spectrum,  derived by taking the ratio of the raw spectrum to the SG filter 
and subtracting unity from the ratio. }
  \label{fig:raw_sg_power}
\end{figure}

\begin{figure} [htbp]
  \centering
  \includegraphics[width=8.6cm]{figures/RescaledPower_vs_Freq_Step_0100.pdf}
  \caption{
  The rescaled power spectrum, obtained by multiplying the normalized 
power with the ratio of the system noise to the expected axion signal power, 
with the Lorentzian response of the cavity taken into account.}
  \label{fig:rescaled_power}
\end{figure}



%First we will choose a window and a order, then move the window and fit the data a with a polynomial with chosen order, it is a kind of generalization  moving average characterized, if the windows is too big, then it will not remove small structures, if it too small, it may kill the signal, you can see that choosing an appropriate window is important, a way to test if the windows are appropriate is to see the system temperature calculated by  $\mu$ and $\sigma$.

%\begin{equation}
%    \label{eq:ts_mu}
%    \mu = k_{B} \cdot T_{S} \cdot \Delta \nu 
%\end{equation}
%\begin{equation}
%    \label{eq:ts_sigma}
%    \sigma = \frac{k_{B} \cdot T_{S} \cdot \Delta \nu}{\sqrt{N}}
%\end{equation}

%If we treat the spectrum after the SG filter as a pure noise, we know that the system temperature of a white noise can be calculated by $\mu$ and $\sigma$ (Eq.\eqref{eq:ts_mu}) (Eq.\eqref{eq:ts_sigma}),
%where $k_{B}$ is the Boltzmann constant, $T_{S}$ is the system temperature, $\Delta \nu$ is the frequency resolution and N is the number of averaging. \\
%If the chosen window is appropriate, the system temperature estimated from $\mu$ and $\sigma$ should be consistent  with each other. \\

%We also did some studies to check if the SG filter removed the axion signal. Assuming a signal bandwidth of 5 kHz, we added the signal into noise spectrum and applied the SG filter. The result shows that the filter does not suppress the axion signal as given in Fig.\ref{fig:weighted_snr}.

%about whether the sg filter will remove the Axion signal, assuming the Axion signal bandwidth is 5KHz, add the signal in a white noise and apply the sg filter , the results show that it will not affect much before and after use, after divide the sg filter result, we will subtract 1 to make the value become 1.

%\begin{figure}[h]
%    \begin{minipage}[h]{.5\textwidth}
%    \centering
%    \includegraphics[width=0.8\textwidth, height = 0.5\textwidth]{Figure/sg_simulation.png}
%    \caption{The simulation for testing the effect of SG filter.}
%    \label{fig:weighted_snr}
%%    \end{minipage}%
%\end{figure}
%\begin{figure}[h]
%    \begin{minipage}[t]{.5\textwidth}
%    \centering
%    \includegraphics[width=0.8\textwidth,height = 0.5\textwidth]{Figure/weighted_snr.png}
%    \caption{Weighed SNR}
%    \label{fig:weighted_snr}
%%    \end{minipage}%
%\end{figure}

\subsection{Combine the spectra with the weighting algorithm} 
\label{sec:weighting_algorithm}

The purpose of the weighting algorithm is to add different spectra vertically,
 particularly for the frequency bins that appear in multiple spectra.  
Each spectrum was collected with a different cavity resonant frequency. 
Therefore, if a signal appears in a certain frequency bin $j$, due to the
 difference in the resonant frequency and the Lorentzian response, the 
expected signal
 power will be different in each spectrum $i$. The weighting algorithm is 
expected to take this into account with a weight calculated for each bin $j$ of
 the normalized and rescaled spectrum $i$, as defined in Eq.~\eqref{eq:weight}.
The weighted power $\delta^\text{com}_{n}$ and the standard deviation 
$\sigma^\text{com}_{n}$ of the $n^\text{th}$ bin in the combined spectrum are 
calculated using Eq.~\eqref{eq:comb_power} and Eq.~\eqref{eq:comb_sigma}, 
respectively. The SNR$^\text{com}_{n}$ is the ratio of 
$\delta^\text{com}_{n}$ to 
$\sigma^\text{com}_{n}$ as given in Eq.~\eqref{eq:comb_snr}. 
Figure~\ref{fig:power_sigma_comb} and Fig.~\ref{fig:SNR_comb} show the power, 
the standard deviation, and the SNR of the combined spectrum, respectively.


\begin{equation}
    \label{eq:weight}
    %    {w_{n}}^{i} = \frac{h \cdot p}{(\sigma_{n}^{i})^{2}}
    {w_{ij}} = \frac{1}{(\sigma_{ij}^\text{res})^{2}},
\end{equation}

\begin{equation}
    \label{eq:comb_power}
    \delta_{n}^\text{com} = \frac{ \sum_{1}^{k}\left(\delta_{ij}^\text{res} \cdot {w_{ij}}\right)}{\sum_{1}^{k} {w_{ij}}},
\end{equation}
\begin{equation}
    \label{eq:comb_sigma}
    \sigma_{n}^\text{com} = \frac{ \sqrt{\sum_{1}^{k}(\sigma_{ij}^\text{res} \cdot {w_{ij}})^2}}{\sum_{1}^{k} {w_{ij}}},
\end{equation}
\begin{equation}
    \label{eq:comb_snr}
    \text{SNR}_{n}^\text{com} = \frac{\delta^\text{com}_{n}}{\sigma^\text{com}_{n}}= \frac{\sum_{1}^{k}\left(\delta_{ij}^{res} \cdot {w_{ij}}\right)}{ \sqrt{\sum_{1}^{k}(\sigma_{ij}^{res} \cdot {w_{ij}})^2}},
\end{equation}
with $i$ running from 1 to $k$ where $k$ is the 
number of spectra that share the same frequency bin $j$.

\begin{figure}[h]
    \centering
    \includegraphics[width=8.6cm]{figures/Power_CombSpectrum_AxionRun_AllSteps_Rescan_SG4_W201_LqWeight.png}
    \includegraphics[width=8.6cm]{figures/Sigma_CombSpectrum_AxionRun_AllSteps_Rescan_SG4_W201_LqWeight.png}
    \caption{The combined power $\delta$ following Eq.~\eqref{eq:comb_power} 
(upper) and the standard deviation $\sigma$ derived from 
Eq.~\eqref{eq:comb_sigma} (lower).}
    \label{fig:power_sigma_comb}
\end{figure}

\begin{figure}[hbt!]
    \centering
    \includegraphics[width=8.6cm]{figures/SNR_CombSpectrum_AxionRun_AllSteps_Rescan_SG4_W201_LqWeight.png}
    \caption{The signal-to-noise ratio (SNR) calculated using 
Eq.\eqref{eq:comb_snr} of the combined spectrum. }
    \label{fig:SNR_comb}
\end{figure}


%where ${\delta_n^i}$ and ${\sigma_n^i}$ are the measured power and the corresponding standard deviation of the ${n^{\mathrm{th}}}$ frequency bin of the ${i^{\mathrm{th}}}$ spectrum., From the weighted power in ${n^{th}}$ bin in Eq.\eqref{eq:weighted_power}, and weighted $\sigma$ in Eq.\eqref{eq:weighted_sigma}, we can get our Signal to noise ratio as Eq.\eqref{eq:weighted_SNR}

\subsection{Merge bins}
\label{sec:merge}

The expected axion bandwidth is about 5~kHz at the frequency of 5~GHz. In 
this paper, the interested frequency range is \flo -- \fhi~GHz and the bin 
width is 1~kHz. Therefore, in order to maximize the SNR, five consecutive 
bins with overlapping of the combined spectrum are merged to construct a 
final spectrum.
The purpose of overlapping is to avoid the signal power broken into different 
neighboring bins of the merged spectrum. Before defining the weights for 
merging, 
the power and the standard deviation of each bin in the combined spectrum are 
multiplied with $M=5$: $\delta^{c}_n \rightarrow M\delta^\text{com}_n$ and 
$\sigma^{c}_n \rightarrow M \sigma^\text{com}_n$. This rescaling gives the 
expected mean of the normalized power $\mu^\text{com}_k = 1$ if a KSVZ axion 
signal power leaves a fraction 1/$M$ of its power in the $k^\text{th}$ 
bin of the combined spectrum.
Then the maximum likelihood weights, defined in Eq.~\eqref{eq:merge_weight} 
based on the Maxwellian line shape for axions [Eq.~\eqref{eq:simplesignal}], 
are used to build the merged spectrum. 



\begin{equation}
    \label{eq:merge_weight}
    w_{n} = \frac{L_{n}}{(\sigma_{n}^{c})^{2}} = \frac{L_{n}}{(M\sigma_{n}^\text{com})^{2}},
\end{equation}
where $M = 5$ is the number of merged bins, and 

%\begin{equation}
%    \label{eq:axion_line_shape}
%    f(\nu) = \frac{2}{\sqrt{\pi}}\sqrt{\nu - \nu_a} \left( \frac{3}{\nu_a \big \langle \beta^{2} \big \rangle }\right)^{\frac{3}{2}} e^{- \frac{3(\nu-\nu_a)}{\nu_a \big \langle \beta^{2} \big \rangle}}
%\end{equation}

\begin{equation}
    \label{eq:Lq_integtal}
    L_{n} = M \int_{\nu_a +\delta\nu_m + (n-1)\Delta\nu}^{\nu_a +\delta\nu_m + n\Delta\nu} f(\nu) \,d\nu,
\end{equation}
where $n = 1,..,M$, $\nu_a=\left.\ma c^2\middle/h\right.$ is the axion frequency, 
$\delta\nu_m$ is the misalignment between $\nu_a$ and the lower bin 
boundaries in the combined spectrum and $\Delta\nu =1$~kHz is the frequency bin width. 
The function $f(\nu)$ has been defined in Eq.~\eqref{eq:simplesignal}. 

%where $L_q$ is the integral of the line shape from the lower edge to higher 
%edge of ${q^{th}}$ bin. 
The power, the standard deviation and the SNR of the merged spectrum are:

%We have a weight Eq.\eqref{eq:merge_weight}, where Lq is the area in ${q^{th}}$ bin and $\sigma_{q}$ is the weighted $\sigma$ in the $q^{\mathrm{th}}$ bin , Eq.\eqref{eq:axion_line_shape} is the axion CDM(cold dark matter) line shape, where $\big \langle \beta^{2} \big \rangle = \big \langle v^{2} \big \rangle /c^{2}$ and $\big \langle v^{2} \big \rangle = (270km/s)^{2}\ , \big \langle v^{2} \big \rangle$ is the squared virial velocity. Eq.\eqref{eq:L_q_integtal},

\begin{equation}
    \label{eq:merged_power}
    \delta_{g}^\text{merged} = \frac{ \sum_{n = 1}^{M}\left(\delta_{g+n-1}^{c} \cdot {w_{g+n-1}}\right)}{\sum_{n = 1}^{M} {w_{g+n-1}}},
\end{equation}

\begin{equation}
    \label{eq:merged_sigma}
    \sigma_{g}^\text{merged} =  \frac{ \sqrt{\sum_{n = 1}^{M} \left(\sigma_{g+n-1}^{c} \cdot {w_{g+n-1}}\right)^2}}{\sum_{n = 1}^{M} {w_{g+n-1}}},
\end{equation}

\begin{equation}
    \label{eq:merged_snr}
    \text{SNR}_{g}^\text{merged} = \frac{\delta^\text{merged}_{g}}{\sigma^\text{merged}_{g}} = \frac{\sum_{n = 1}^{M}\left(\delta_{g+n-1}^{c} \cdot {w_{g+n-1}}\right)}{ \sqrt{\sum_{n = 1}^{M} \left(\sigma_{g+n-1}^{c} \cdot {w_{g+n-1}}\right)^2}},
\end{equation}
where $g = 1,..,N-M+1$ is the index for the frequency bins in the final spectrum.  
The total number of bins in the combined (final merged) spectrum is $N$ ($N-M+1$). 


%Adding adjacent bin with a weight Eq.\eqref{eq:merged_power} Eq.\eqref{eq:merged_sigma}, where k is the number of adjacent bin to be merged, q is ${q^{th}}$ merged bin, with Eq.\eqref{eq:merged_sigma}, we can get our weighted merged power spectrum FIG.\ref{fig:merged_data}.


\subsection{Rescan and set limits on \gagg} 
Before the collection of the CD102 data, a 5$\sigma$ SNR target was chosen, 
which corresponds to a candidate threshold of 3.355$\sigma$ at 95\% confidence.
 After the merging as described in Sec.~\ref{sec:merge}, if there were 
any potential signal with an SNR larger than 
3.355$\sigma$, a rescan would be proceeded to check if it were a real signal 
or a statistical fluctuation. 
The procedure of the CD102 data taking was to perform a rescan after 
covering every 10~MHz; the rescan was done by adjusting the tuning rod of the 
cavity so to match the resonant frequency to the frequency of the candidate. 
In total, 22 candidates with an SNR greater than 3.355$\sigma$ were found. 
Among them, 17 candidates were from the fluctuations because they were gone 
after a few rescans. 
The remaining five candidates reached an SNR greater than $4\sigma$ after 
rescanning; a 
portable antenna outside the DR was used to probe if they came from external 
sources. 
The external signals in the frequency ranges of 4.710170 -- 4.710190~GHz and 
4.747301 -- 4.747380~GHz from the instruments in the laboratory were detected, 
therefore, no limits are placed for these two ranges.  
More details can be found in the 
TASEH instrumentation paper~\cite{TASEHInstrumentation}. 
Figure~\ref{fig:power_sigma_merged} and Fig.~\ref{fig:SNR_merged} show the 
power, the standard deviation, and the SNR of the merged spectrum after 
including data from both the original scans and the rescans, respectively. 

Since no candidates were found after the rescan, an upper limit on 
the signal power $P_s$ is derived by setting $P_s$ equal to 
$5\sigma_{q}^\text{merged}$ for a certain frequency bin $q$ in the merged 
spectrum.  Then, the 95\% C.L. limits on the dimensionless parameter 
\ggamma\ and the axion-two-photon coupling \gagg\ could be derived 
according to Eq.~\eqref{eq:ps} and Eq.~\eqref{eq:grelation}. 
See Sec.~\ref{sec:results} for the final limits including the systematic 
uncertainties.

\begin{figure}[h]
    \centering
    \includegraphics[width=8.6cm]{figures/Power_GrandSpectrum_AxionRun_AllSteps_Rescan_Merged_5bin_SG4_W201_LqWeight.png}
    \includegraphics[width=8.6cm]{figures/Sigma_GrandSpectrum_AxionRun_AllSteps_Rescan_Merged_5bin_SG4_W201_LqWeight.png}
    \caption{The merged power $\delta$ following Eq.~\eqref{eq:merged_power} 
(upper) and the standard deviation $\sigma$ derived from Eq.~\eqref{eq:merged_sigma} (lower). The results shown were obtained using data from both the 
original scans and the rescans.}
    \label{fig:power_sigma_merged}
\end{figure}

\begin{figure}[hbt!]
    \centering
    \includegraphics[width=8.6cm]{figures/SNR_GrandSpectrum_AxionRun_AllSteps_Rescan_Merged_5bin_SG4_W201_LqWeight.png}
    \caption{The signal-to-noise ratio (SNR) calculated using Eq.~\eqref{eq:merged_snr} for the merged spectrum including data from both the original 
scans and the rescans. No candidate exceeds the threshold of 
$3.355\sigma$ (solid-black horizontal line). }
    \label{fig:SNR_merged}
\end{figure}

\section{Analysis of the Synthetic Axion Data}\label{sec:faxion}
After TASEH finished collecting the CD102 data on November 15, 2021, 
the synthetic axion signals were injected into the cavity and read out via the 
same transmission line and amplification chain. The procedure 
to generate axion-like signals is summarized in 
Ref.~\cite{TASEHInstrumentation}. 
Due to the uncertainties on the losses of readout electronics and transmission
 lines, the synthetic axion signals were not used to perform an absolute 
calibration of the search sensitivity. Instead, 
a test with synthetic axion signals could be used to verify the procedures of 
data acquisition and physics analysis. The 
signal-to-noise ratio (SNR) of the frequency bin with maximum power from the 
synthetic axion signals, at 4.708970~GHz, was set to $\approx 3.35\sigma$, 
corresponding to a power of $\approx 6.03 \times 10^{-13}$~W in a 1-kHz 
frequency bin.  

The same analysis procedure as described in Section~\ref{sec:ana} was applied 
to the data with synthetic axion signals. 
Figure~\ref{fig:faxionstep} presents the individual raw power spectra in 
24 frequency scans. Before combining 
the 24 spectra vertically, the SNR of the maximum-power bin is measured to be 
3.577$\sigma$; the SNR is slightly higher than 3.35$\sigma$ due to a 
5\% difference in the noise fluctuation between the measurements from 
the calibration and the measurements taken 
right before injecting axion-like signals. After the vertical combination 
of power spectra and the merging of five frequency bins, the SNRs increase to 
4.74$\sigma$ and 6.12$\sigma$, respectively. In addition to the 
injected synthetic axion signal, a candidate at 4.708006~GHz was found after 
merging the spectra. Since it was not possible to perform a rescan later, 
the real axion data from the two scans that had resonant frequencies close to 
the candidate frequency were added so to mimic the rescan; the candidate 
 disappeared afterwards and is a statistical fluctuation.  
Figures~\ref{fig:faxioncombine}--~\ref{fig:faxionmerge} present 
the spectra with the corresponding SNR, respectively, after combining the 24 
spectra vertically and after merging five neighboring bins, 
including both the 24 scans of the synthetic axion data and the two scans 
of the real axion data. 
The analysis results of the synthetic axion signals prove that an power 
excess of more than 5$\sigma$ can be found at the expected frequencies via 
the standard analysis procedure.  

\begin{figure}[htbp]                                                                                                  
    \centering                                                                                                                       
%    \includegraphics[width=0.48\textwidth]{figures/RawSpectra_Faxion_YAxis_Shifted.pdf}
    \includegraphics[width=8.6cm]{figures/faxion_rawpower_24steps.png}
 \caption{The raw output power spectra, before applying the 
 SG filter, from the 24 frequency steps of the synthetic axion 
data. In order to show the spectra clearly, the spectra are shifted 
with respect to each other with an arbitrary offset in the vertical scale.}                
\label{fig:faxionstep}                                                                                                            
\end{figure}                       

\begin{figure}[htbp]                                                                                                  
    \centering                                                                                                                       
    \includegraphics[width=8.6cm]{figures/Power_CombSpectrum_FaxionRun_AllSteps_Rescan_SG4_W201.png}
    \includegraphics[width=8.6cm]{figures/SNR_CombSpectrum_FaxionRun_AllSteps_Rescan_SG4_W201.png}
    \caption{The power (upper) and the 
signal-to-noise ratio (lower) after combining the spectra 
of the synthetic axion data
with overlapping 
frequencies vertically. The procedure and the weights for combination 
are summarized in Section~\ref{sec:ana}.}                
\label{fig:faxioncombine}                                                                                                            
\end{figure}                       


\begin{figure}[htbp]                                                                                                  
    \centering                                                                                                                       
    \includegraphics[width=8.6cm]{figures/Power_GrandSpectrum_FaxionRun_AllSteps_Rescan_Merged_5bin_SG4_W201_LqWeight.png}
    \includegraphics[width=8.6cm]{figures/SNR_GrandSpectrum_FaxionRun_AllSteps_Rescan_Merged_5bin_SG4_W201_LqWeight.png}
    \caption{The power (upper) and the signal-to-noise ratio (lower) after 
merging the power measured in five neighboring frequency bins of the 
synthetic axion data. 
The procedure and the weights for merging 
are summarized in Section~\ref{sec:ana}.}                
\label{fig:faxionmerge}    
\end{figure}                       

   

\section{Systematic Uncertainties} \label{sec:sys}
The systematic uncertainties on the \gagg\ limits arise from the 
following sources:
\begin{itemize}
\item Uncertainty on the product 
$\left.Q_L\beta\middle/\left(1+\beta\right)\right.$ in Eq.~\eqref{eq:ps}: 
In order to extract the loaded quality factor $Q_L$ and the coupling 
coefficient $\beta$, a fitting of the measured results of the cavity 
scattering matrix was performed. A relative uncertainty of 3.6\% is 
assigned to this product, after a comparison of the measurements at 
$T_\text{c}\simeq155$~mK with a prediction rescaled from the measurements 
at room temperature. More details about the measurements of the 
cavity properties can be found in Ref.~\cite{TASEHInstrumentation}. 
A 3.6\% variation of this product results in a 1.9\% uncertainty 
on the \gagg\ limits. 

\item Uncertainty on the form factor $C_{010}$: 
the variation of $C_{010}$, due to the different grid sizes in the integrals 
of Eq.~\eqref{eq:formfactor}, is within 1\%, which gives a $\leq 0.5$\% 
uncertainty on the \gagg\ limits.

\item Uncertainties on the noise temperature \ta\ from: (i) the RMS of 
the measurements in the calibration: 
$\left. \Delta \ta\middle/\ta\right.= 2.3\%$,  
%(see Sec.~\ref{sec:calibration} and Fig.~\ref{fig:hemtcalvsf}).
%
%\item Uncertainty on the noise temperature \ta\ 
and (ii) from the largest difference 
between the value determined by the calibration and that from the CD102 
data: $\left. \Delta \ta\middle/\ta\right.= 4\%$ 
(see Sec.~\ref{sec:calibration} and Fig.~\ref{fig:hemtcalvsf}). 
These two uncertainties on the \ta\ result in a 2.8\% uncertainty 
on the \gagg\ limits. 

%\item Uncertainty on the misalignment $\delta f_m$ between the true 
%axion frequency $f_a$ and the lower bin boundaries in the merged spectrum 
%(see Sec.~\ref{sec:merge}).
\item Uncertainty due to the misalignment (see Sec.~\ref{sec:merge}):
  estimated by comparing the central results to the one without misalignment
  ($\delta f_m = 0$)
  and to the ones with given values of $\delta f_m$.
  The comparison shows that $\delta f_m = 0$ gives the largest difference 
  of 2.8\% on the limit, which is used as the systematic uncertainty from the 
  misalignment.
  
\item Uncertainty from the choice of the SG-filter parameters: i.e.  
the window width and the order of the polynomial in the SG filter. At the 
beginning of the data taking, a preliminary optimization was performed: a 
window width of 201 bins and a 4$^\text{th}$-order polynomial were used for 
the first analysis of the CD102 data (see Sec.~\ref{sec:ana}). 
This choice is kept for the central results. 
Nevertheless, various methods of optimization are also explored. The goal 
of the optimization is to find a set of SG-filter parameters that only 
model the noise spectrum and do not remove a real signal. 
The methods include:
\begin{itemize}
 \item Minimize the difference between the two outputs returned by the SG 
filter, when the SG filter is applied to: (i) the real data only, and (ii) 
the sum of the real data and the simulated axion signals. 
 \item Minimize the difference between the output returned by the 
 SG filter and the function ${\cal G}_\text{noise}$ 
that models the noise spectrum (derived by fitting the CD102 data), 
when the SG filter is applied to the sum of the simulated noise based on 
${\cal G}_\text{noise}$ and the simulated axion signals. 
See Fig.~\ref{fig:sgcompare} for an example of the 
simulated spectrum, the function ${\cal G}_\text{noise}$, and the 
output returned by 
 the SG filter when a 3$^\text{rd}$-order polynomial and a window of 141 
 bins are chosen; the squared differences from all the frequency bins are 
summed together (rescaled $\chi^2$) when performing the optimization.
 Figure~\ref{fig:sgoptimize} shows the rescaled $\chi^2$ 
as a function of window widths when the order of polynomial is 
 set to three, four, and six. 
 \item Compare the mean $\mu_\text{noise}$ and the width $\sigma_\text{noise}$ 
of the measured power after applying the SG filter, 
assuming that no signal is present in the 
data. See Fig.~\ref{fig:noisegauss} for an example distribution 
of the measured power from the averaged spectrum of a 
single scan; %, when the cavity resonant frequency 
%is 4.798147~GHz; 
a Gaussian fit is performed to extract 
$\mu_\text{noise}$ and $\sigma_\text{noise}$. Given the nature of the 
thermal noise~\cite{Dicke}, the two variables are supposed to be related to 
each other if a proper window width and a proper order are chosen:
\begin{equation*} 
\sigma_\text{noise} = \frac{\mu_\text{noise}}{\sqrt{N_\text{spectra}}},
\end{equation*}
where $N_\text{spectra}$ is the number of spectra for averaging and 
is related to the amount of integration time for each frequency step. In 
general, $N_\text{spectra}=1920000-2520000$. 
\end{itemize}

In addition, one could choose to optimize for each frequency step 
individually, optimize for a certain frequency step but apply the results to 
all data, or optimize by fitting together the spectra from all the frequency 
steps. %Figure~\ref{fig:syssgfilter} shows that 
The deviations from the central results using different optimization 
approaches are in general within 1\% and the 
maximum deviation of 1.8\% 
on the \gagg\ limit is used as a conservative estimate of the systematic 
uncertainty from the SG filter. 

\end{itemize}

%The first source of the systematic uncertainty 
%has negligible effect on the limits of \gagg\ while the 
%latter three sources are studied and added in quadrature to obtain the total 
%systematic uncertainty. 
The effects on the \gagg\ limits from these sources are studied and added in 
quadrature to obtain the total systematic uncertainty. 
The systematic uncertainties on the \gagg\ limits 
are displayed together with the central results in Sec.~\ref{sec:results}. 
%Overall the total relative systematic uncertainty is $\approx 3.4\%$.
Overall the total relative systematic uncertainty is $\approx 4.6\%$.

\begin{figure} [htbp]
  \centering
  \includegraphics[width=8.6cm]{figures/GeneratedSpectrum_Optimized_SGFilter_NPar_3_Window_141.pdf}
  \caption{Upper panel: 
 The simulated spectrum (red), including the axion signal and the 
noise, is overlaid with the function that models the noise 
${\cal G}_\text{noise}$ (black) and the 
output returned by the SG filter (green). Lower panel: The ratio of the output 
returned by the SG filter to the function ${\cal G}_\text{noise}$.}
  \label{fig:sgcompare}
\end{figure}


\begin{figure} [htbp]
  \centering
%  \includegraphics[width=0.4\textwidth,height = 0.25\textwidth]{figures/chi2_Different_Order_Window_SGFilter.png}
  \includegraphics[width=8.6cm]{figures/chi2_Different_Order_Window_SGFilter.png}
  \caption{The rescaled $\chi^2$ when various values of window widths and 
  a 3$^\text{rd}$, a 4$^\text{th}$, or a 6$^\text{th}$-order polynomial 
  are applied in the SG filter. The rescaled $\chi^2$ is defined as 
  the sum of the squared differences  
  from all the frequency bins, between the output returned by the SG filter 
  and the function that models the noise spectrum ${\cal G}_\text{noise}$ 
  (see Fig.~\ref{fig:sgcompare}).  
  In this 
  figure, the best choice is a 4$^\text{th}$-order polynomial with 
  a window width of 241 data points (bins). }
  \label{fig:sgoptimize}
\end{figure}
 


\begin{figure} [htbp]
  \centering
  \includegraphics[width=8.6cm]{figures/sysSG_temphistogram.png}
  \caption{An example of the distribution of the measured power after 
applying the SG filter, when 
%the cavity resonant frequency is 4.798147~GHz. The distribution contains 
the cavity resonant frequency is 4.79815~GHz. The distribution contains 
1600 entries and each entry corresponds to the measured power 
in one frequency bin, averaged
over 1920000 subspectra. The mean and the width returned by 
a Gaussian fit to the distribution are used to determine the best choice of 
SG parameters. The fitted Gaussian mean $\mu$ divided by 
$\sqrt{1920000}$ is consistent 
with the fitted Gaussian width $\sigma$. The best choice of SG parameters 
obtained for this scan is a window of 189 data points (bins) with a 
3$^\text{rd}$-order polynomial. 
%The mean $\mu_\text{noise}=3.2\times10^{-20}$~W in 
%a 1-kHz frequency bin would imply a noise temperature of 2.3~K.
}
  \label{fig:noisegauss}
\end{figure}
 

%\begin{figure} [htbp]
%  \centering
%  \includegraphics[width=8.6cm]{figures/sys_compareSG_4_201.png}
%  \caption{The ratios of the limits on \gagg\ due to the different choices 
% of the window width and the order of polynomial in the SG filter, with 
%respect to 
% the central results (a window width of 201 bins and the 4$^\text{th}$-order 
% polynomial). The window width of 241 bins and the 4$^\text{th}$-order 
% polynomial are obtained from the optimization after injecting an axion 
%signal on top of a simulated noise spectrum. The window width of 189 bins and 
%the 3$^\text{rd}$-order polynomial are obtained from the optimization 
% after comparing the means and the widths of the measured power distributions.}
%  \label{fig:syssgfilter}
%\end{figure}
 


\section{Results}
In Fig.\ref{fig:SNR_histogram}, we set a threshold at $3.355\sigma$, if the expected signal is 5 $\sigma$ above the noise level, the signal is supposed to have $95\% $ of the chance passing the cut at 3.355 sigma. However, a signal was not observed. Therefore, with $95{\%}$ C.L., the signal cannot be larger than 5 $\sigma$.

\begin{figure}[hbt!]
    \centering
%    \includegraphics[width=0.5\textwidth,height = 7cm]{Figure/limit_5sigma.png}
    \caption{Set the limit with 5 $\sigma$}
    \label{fig:limit_5sigma}
\end{figure}




\section{Conclusion} 
This paper presents the first results of a search for axions for the mass 
range $\mlo < \ma < \mhi \muevcc$, using the data collected by the Taiwan 
Axion Experiment with Haloscope from October 13, 2021 to November 15, 2021. 
Apart from external signals, no candidates with significance more than
3.355~$\sigma$ were found. The experiment excludes models with the axion-two-photon
coupling $\gagg\gtrsim \avelimit\GeVinv$, a factor of ten above the benchmark
KSVZ model. For the first time, 
constraints on the $\gagg$ have been placed in this mass region.



\begin{acknowledgments}

\end{acknowledgments}

\appendix
\section{The derivation of the noise spectrum from the cavity} 
\label{sec:cavitynoise}

The Hamiltonian of a single-mode cavity is
\begin{equation}
\label{eq:cavity_Hamiltonian}
    H = \hbar \omega_{\rm c}(C^{\dagger}C+\frac{1}{2}),
\end{equation}
where $\omega_{\rm c}/2\pi$ is the cavity resonance frequency and $C$ is the 
annihilation operator of the inner cavity field. The cavity field is coupled 
to the modes $A$ of a transmission line with the rate $\kappa_2$. The cavity 
field is also coupled to the environment modes $B$ with the rate $\kappa_0$. 
Based on the model of Fig.~\ref{fig:cavity_in_out.jpg} and the input-output 
theory, the equation of motion for $C$ is obtained:
\begin{equation}
\label{eq:EOM}
    \frac{dC}{dt} = -i\omega_{\rm c} C - \frac{\kappa_2+\kappa_0}{2} C + \sqrt{\kappa_2} A_{\rm in} + \sqrt{\kappa_0}B_{\rm in}.
\end{equation}
A boundary condition holds for the transmission modes:
\begin{equation}
\label{eq:BC}
    A_{\rm out} = \sqrt{\kappa_2} C - A_{\rm in}.
\end{equation}
Considering working in a rotating frame of the signal frequency $\omega$ near 
$\omega_{\rm c}$, the equation of motion becomes:
\begin{equation}
\label{eq:EOM_rotating_frame}
    -i\omega C + \frac{dC}{dt} = -i \omega_{\rm c} C - \frac{\kappa_2+\kappa_0}{2} C + \sqrt{\kappa_2} A_{\rm in} + \sqrt{\kappa_0} B_{\rm in}.
\end{equation}
The steady state solution for the cavity field is: 
\begin{equation}
\label{eq:cavity_field}
    C = \frac {\sqrt{\kappa_2} A_{\rm in} + \sqrt{\kappa_0} B_{\rm in}} {-i (\omega-\omega_{\rm c}) + \frac{\kappa_2+\kappa_0}{2} }.
\end{equation}
By substituting Eq.~(\ref{eq:cavity_field}) into Eq.~(\ref{eq:BC}), the 
reflected modes of the transmission line $A_{\rm out}$ are expressed in terms 
of the input modes of the transmission line $A_{\rm in}$ and the environment 
$B_{\rm in}$:
\begin{equation} \label{eq:output_field}
\begin{split}
	 A_{\rm out} & =  \frac{i(\omega - \omega_{\rm c}) + \frac{\kappa_2-\kappa_0}{2}}{-i(\omega - \omega_{\rm c}) + \frac{\kappa_2+\kappa_0}{2}} A_{\rm in} + \frac{\sqrt{\kappa_2 \kappa_0}}{-i(\omega - \omega_{\rm c}) + \frac{\kappa_2+\kappa_0}{2}} B_{\rm in} \\
                           & = \frac{-(\omega-\omega_{\rm c})^2 + \frac{\kappa_2^2-\kappa_0^2}{4}+i\kappa_2(\omega-\omega_{\rm c})}{(\omega-\omega_{\rm c})^2+(\frac{\kappa_2+\kappa_0}{2})^2} A_{\rm in} + \frac{\sqrt{\kappa_2\kappa_0}\frac{\kappa_2+\kappa_0}{2}+i\sqrt{\kappa_2\kappa_0}(\omega-\omega_{\rm c})}{(\omega-\omega_{\rm c})^2+(\frac{\kappa_2+\kappa_0}{2})^2} B_{\rm in}.
\end{split}
\end{equation}

Therefore, the autocorrelation of $A_{\rm out}$ is related to those of $A_{\rm in}$ and $B_{\rm in}$:
\begin{equation}
\label{eq:autocorrelation}
\begin{split}
    \langle A_{\rm out}^\dagger A_{\rm out} \rangle  = & \frac{[(\omega-\omega_{\rm c})^2 - \frac{\kappa_2^2-\kappa_0^2}{4}]^2+\kappa_2^2(\omega-\omega_{\rm c})^2}{[(\omega-\omega_{\rm c})^2+(\frac{\kappa_2+\kappa_0}{2})^2]^2} \langle A_{\rm in}^\dagger A_{\rm in} \rangle \\ & +  \frac{\kappa_2\kappa_0(\frac{\kappa_2+\kappa_0}{2})^2+\kappa_2\kappa_0(\omega-\omega_{\rm c})^2}{[(\omega-\omega_{\rm c})^2+(\frac{\kappa_2+\kappa_0}{2})^2]^2}\langle B_{\rm in}^\dagger B_{\rm in} \rangle.
\end{split}
\end{equation}
The spectrum from the cavity $S(\omega)$ is found to be related to the spectrum 
of the readout transmission line $S_{\rm rt}(\omega)$ and the spectrum of the 
cavity environment $S_{\rm cav}(\omega)$:
\begin{equation}
\label{eq:spectrum_relation}
\begin{split}
    S(\omega) & = \frac{[(\omega-\omega_{\rm c})^2 - \frac{\kappa_2^2-\kappa_0^2}{4}]^2+\kappa_2^2(\omega-\omega_{\rm c})^2}{[(\omega-\omega_{\rm c})^2+(\frac{\kappa_2+\kappa_0}{2})^2]^2} S_{\rm rt}(\omega) \\ & + \frac{\kappa_2\kappa_0(\frac{\kappa_2+\kappa_0}{2})^2+\kappa_2\kappa_0(\omega-\omega_{\rm c})^2}{[(\omega-\omega_{\rm c})^2+(\frac{\kappa_2+\kappa_0}{2})^2]^2}S_{\rm cav}(\omega).
\end{split}
\end{equation}
As the the readout transmission line and the cavity environment are both in 
thermal states, i.e. $S_{\rm rt}(\omega)=\left[n_{\rm BE}(T_{\rm rt})+1/2\right]\hbar\omega$ 
and $S_{\rm cav}(\omega)=\left[n_{\rm BE}(T_{\rm cav})+1/2\right]\hbar\omega$, where 
$n_{\rm BE}$ is the mean photon number given by the Bose-Einstein distribution, 
$S(\omega)$ is white if $T_{\rm cav}=T_{\rm rt}$, and Lorentzian if 
$T_{\rm cav} \gg T_{\rm rt}$.

\begin{figure}[htbp]
    \centering
    \includegraphics[width=0.48\textwidth]{figures/cavity_in_out.jpg}
    \caption{A cavity is coupled to the modes of transmission line $A$ with the 
rate $\kappa_2$ and the modes of environment $B$ with the rate $\kappa_0$.}
    \label{fig:cavity_in_out.jpg}
\end{figure}




%\section{Appendixes}

\bibliography{main}% Produces the bibliography via BibTeX.

\end{document}
%
% ****** End of file apssamp.tex ******
