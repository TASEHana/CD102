
\section{Experimental Setup}\label{sec:taseh} 
The detector of TASEH is located at the Department of Physics, National 
Central University, Taiwan and housed within a cryogen-free dilution 
refrigerator (DR) from BlueForcs. A 8-Tesla superconducting solenoid with a 
bore diameter of 76~mm and a length of 240 mm is integrated with the DR. 

The data for the analysis presented in this paper were collected by TASEH 
from October 13, 2021 to November 15, 2021, and termed as the CD102 data, 
where CD stands for ``cool down''. 
During the data taking, the cavity sat in the center of the magnet bore 
and was connected via holders 
to the mixing chamber plate of the DR at a temperature of $\approx$30~mK. 
Due to an inefficiency of thermal conduction and thermal radiation, 
the temperature of the cavity stayed at 155~mK, higher with respect to the DR.
The cavity, made of oxygen-free high-conductivity (OFHC) copper, has an 
effective volume of 0.234~L and is a two-cell cylinder split along 
the axial direction ($z$-axis). 
The cylindrical cavity has an inner radius of 2.5~cm and a 
height of 12~cm.  In order to maintain a smooth surface, the cavity underwent 
the processes of polishing, chemical cleaning, and annealing. The resonant 
frequency of the TM$_{010}$ mode can be tuned over the range of 
4.717--4.999~GHz via the rotation of an off-axis OFHC copper tuning rod, from 
the position closer to the cavity wall to the position closer to the cavity 
center (i.e. when the vector from the rotation axis to the tuning rod is 
at an angle of $0^\circ$ to $180^\circ$, with respect to the vector from the 
cavity center to the rotation axis). 
The CD102 data cover the frequency range of \flo--\fhi~GHz. 
There were 839 resonant-frequency steps in total, with a frequency difference of 
95--115~kHz between the steps. Each resonant-frequency step is denoted as a ``scan'' 
and the data integration time was about 32-42 minutes. The integration 
time was determined based on the target \gagg\ limits and the experimental 
parameters in Table~\ref{tab:tasehbenchmark}. The form factor $C_{010}$ as 
defined in Eq.~\eqref{eq:formfactor} varies from 0.64 to 0.69 over the 
full frequency range.  
The intrinsic, unloaded quality factor $Q_0$ at the cryogenic temperature 
($T_\mathrm{cavity}\simeq 155$~mK) is $\simeq 60000$ at the frequency of 
4.74~GHz.

An output probe, made of a 50-$\Omega$ semi-rigid coaxial cable 
soldering SMA (SubMiniature version A) plug crimp, was inserted into the 
cavity and its depth was set for 
$\beta\simeq2$.  The signal from the output probe was directed to an 
impedance-matched amplification chain. The first-stage amplifier was 
a low noise high-electron-mobility transistor (HEMT) amplifier with an 
effective noise temperature of $\approx 2$~K, mounted on the 4K-flange. 
The signal was further amplified at room temperature via a 
three-stage post-amplifier, and down-converted 
and demodulated to in-phase (I) and quadrature (Q) components and digitized 
by an analog-to-digital converter (ADC) with a sampling rate of 2~MHz. 
%The frequency resolution of the spectra was 1~kHz.

A more detailed description of the TASEH detector, the operation of the 
data run, and the calibration of the gain and added noise temperature of the 
whole amplification chain can be found in Ref.~\cite{TASEHInstrumentation}. 
See Table~\ref{tab:tasehbenchmark} for the benchmark experimental parameters 
that can be used to estimate the sensitivity of TASEH. 

\begin{table}
\caption{The benchmark experimental parameters for estimating the sensitivity 
of TASEH. The definitions of the parameters can be found in Section~\ref{sec:intro}. 
More details regarding the determination and the measurements of 
some of the parameters may be found in Ref.~\cite{TASEHInstrumentation}.} 
\label{tab:tasehbenchmark}
\begin{center}
\begin{tabular}{cr}
\hline\hline
 $\nu_\mathrm{lo}$ & \flo~GHz\\
 $\nu_\mathrm{hi}$ & \fhi~GHz \\
 $N_\text{step}$ & 839 \\
 $\Delta \nu_\text{step}$ & 95 -- 115 kHz \\
% $\nu_\text{resolution}$ & 1 kHz \\
 $B_0$  & 8 Tesla \\
 $V$ & 0.234 L \\ %234255~mm$^3$ 
 % $C_{010}$ & 0.65 \\
 % $Q_0$ & 60000 \\
 % $\beta$ & 2 \\
 $C_{010}$ & 0.64 -- 0.69 \\
 $Q_0$ & 59000 -- 65000 \\
 $\beta$ & 1.9 -- 2.3 \\
 $T_\mathrm{cavity}$ & 155~mK \\
 \ta & \noise~K \\
 $\Delta \nu_a$ & 5 kHz \\
\hline\hline
\end{tabular}
\end{center}
\end{table}
