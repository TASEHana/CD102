\section{Systematic Uncertainties}
The systematic uncertainties on the \gagg\ limits arise from the 
following sources:
\begin{itemize}
\item Uncertainty on the product 
$\left.Q_L\beta\middle/\left(1+\beta\right)\right.$ in Eq.~(\ref{eq:ps}): 
In order to extract the loaded quality factor $Q_L$ and the coupling parameter 
$\beta$, a fitting of the measured results of the cavity scattering matrix 
was performed, which results in a relative uncertainty of 0.2\% on this 
product. 

\item Uncertainty on the noise temperature \ta\ from the RMS of 
the measurements in the HEMT calibration: 
$\left. \Delta \ta\middle/\ta\right.= 2.3\%$ 
(see Section~\ref{sec:hemtcalibration} and Fig.~\ref{fig:hemtcalvsf}).

\item Uncertainty on the noise temperature \ta\ from the largest difference 
between the value determined by the HEMT calibration and that from the axion 
data: $\left. \Delta \ta\middle/\ta\right.= 4\%$ 
(see Section~\ref{sec:hemtcalibration} and Fig.~\ref{fig:hemtcalvsf}). 

\item Uncertainty from the choice of the SG-filter parameters: i.e.  
the window width and the order of the polynomial in the SG filter. Before 
collecting the axion data, a preliminary optimization was performed: a 
window width of 201 bins and a 4$^\text{th}$ order polynomial were used for 
the first analysis of the CD102 data (see Section~\ref{sec:ana}). 
This choice was kept for the central results. 
Nevertheless, various methods of optimization were also explored. The 
methods include:
\begin{itemize}
 \item Minimize the difference between the two functions returned by the SG 
filter, with and without injecting a simulated axion signal to the real data.
 \item Minimize the difference between the function returned by the 
 SG filter and the input noise function by simulating both the noise 
spectrum (including the Lorentzian distribution due to the cavity noise) 
and the axion signal. See Fig.~\ref{fig:sgcompare} for a comparison of 
 the simulated spectrum, input noise function, and the function returned by 
 the SG-filter when a 3$^\text{rd}$-order polynomial and a window of 141 
 bins are chosen; the differences from all the frequency bins are summed 
 together when performing the optimization.
 Figure~\ref{fig:sgoptimize} shows the difference 
as a function of window widths when the order of polynomial is 
 set to three, four, and six. 
 \item Compare the mean $\mu_\text{noise}$ and the fluctuation of the 
measured power $\sigma_\text{noise}$, assuming no signal is present in the 
data. See Fig.~\ref{fig:noisegauss} for an example distribution 
of the measured power from yyyyy bins when the cavity resonance frequency 
is set to xxx~GHz; a Gaussian fit is performed to extract $\mu_\text{noise}$ 
and $\sigma_\text{noise}$. Given the nature of the 
thermal noise, the two variables are supposed to be related to 
each other if proper window width and order are chosen:
\begin{equation*} 
\sigma_\text{noise} = \frac{\mu_\text{noise}}{\sqrt{N_\text{spectra}}},
\end{equation*}
where $N_\text{spectra}$ is the number of spectra for averaging and 
is related to the amount of integration time for each frequency step. In 
general, $N_\text{spectra}\approx xxxx$. 
\end{itemize}

In addition, one could choose to optimize for each frequency step 
individually, optimize for a certain frequency step but apply the results to 
all data, or optimize by adding all the frequency steps together. 
Figure~\ref{fig:syssgfilter} shows that 
the deviations from the central results using different optimization 
approaches are in general within 1\% and the 
maximum deviation of 1.8\% 
on the \gagg\ limit is used as a conservative estimate of the systematic 
uncertainty from the SG filter. 

\end{itemize}

The first source has negligible effect on the limits of \gagg while the 
latter three sources are studied and added in quadrature to obtain the total 
systematic uncertainty. 

\begin{figure} [htbp]
  \centering
%  \includegraphics[width=0.4\textwidth,height = 0.25\textwidth]{figures/}
  \caption{Upper panel: 
 The simulated spectrum, including the axion signal and 
 the noise from the cavity and the receiver chain, 
 is overlaid with the input noise function and the function returned by 
 the SG filter. Lower panel: The ratio of the function returned by the SG 
 filter to the input noise function.}
  \label{fig:sgcompare}
\end{figure}


\begin{figure} [htbp]
  \centering
%  \includegraphics[width=0.4\textwidth,height = 0.25\textwidth]{figures/}
  \caption{The difference between the function returned by the SG filter 
  and the input noise function, when various values of window widths and 
  a 3$^\text{rd}$, a 4$^\text{th}$, or a 
  6$^\text{th}$-order polynomial are applied in the SG filter. In this 
  example, the best choice is a 4$^\text{th}$-order polynomial with 
  a window width of 241 data points (bins). }
  \label{fig:sgoptimize}
\end{figure}
 


\begin{figure} [htbp]
  \centering
%  \includegraphics[width=0.4\textwidth,height = 0.25\textwidth]{figures/}
  \caption{An example of the distribution of the measured power when 
the cavity resonance frequency is xxxx~GHz. The distribution contains 
xxxx entries and each entry corresponds to the measured power, averaged
over yyy spectra, in one frequency bin. The mean and the width returned by 
a Gaussian fit to the distribution are used to determine the best choice of 
SG parameters.}
  \label{fig:noisegauss}
\end{figure}
 

\begin{figure} [htbp]
  \centering
%  \includegraphics[width=0.4\textwidth,height = 0.25\textwidth]{figures/}
  \caption{The ratio of the limits on \gagg\ due to the different choices 
 of window width and order of polynomial in the SG filter, with respect to 
 the central results (a window width of 201 bins and 4$^\text{th}$-order 
 polynomial). }
  \label{fig:syssgfilter}
\end{figure}
 

