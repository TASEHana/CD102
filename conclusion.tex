
\section{Conclusion} \label{sec:conclusion}
This paper presents the first results of a search for axions for the mass 
range $\mlo < \ma < \mhi \muevcc$, using the CD102 data collected by the 
Taiwan Axion Experiment with Haloscope from October 13, 2021 to November 15, 
2021. 
Apart from the external signals, no candidates with a significance more than
3.355$\sigma$ were found. The experiment excludes models with the 
axion-two-photon coupling $\gagg\gtrsim \avelimit\GeVinv$, a factor of ten 
above the benchmark KSVZ model. This is the first time that 
constraints on the $\gagg$ are placed in this mass region. The synthetic 
axion signals were injected after the collection of data and the 
successful results validated the data acquisition and the analysis procedure. 

The target of TASEH is to search for axions for the mass range of 
16--40\muevcc, corresponding to a frequency range of 3.9--9.7~GHz. 
In the coming years, several upgrades are expected, including: the use of 
the Josephson parametric amplifier as the first-stage amplifier, the 
replacement of the existing dilution refrigerator with a new one that has 
a magnetic field of 9~Tesla and a larger bore size, and the development of 
a new cavity with an effective volume reaching one liter. These upgrades 
will reduce the added noise by a factor of 10 and increase the magnetic 
field and the cavity volume by a factor of 1.125 and 5. With the 
improvements of the experimental setup and several years of data taking, 
TASEH is expected to probe the QCD axion photon band in the target mass range.


